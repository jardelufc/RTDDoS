\chapter[Conclusões e Trabalhos Futuros]{Conclusões e Trabalhos Futuros}
% ----------------------------------------------------------
Esse trabalho apresentou um estudo de um \textit{framework} de detecção de ataques DDoS, o qual recebe como entrada um arquivo contendo um dataset e analisa todos os tráfegos, fornecendo como resposta se o tráfego analisado é normal ou malicioso. O \textit{framework} foi implementado na linguagem MATLAB, o qual recebe um arquivo de entrada contendo o tráfego a ser analisado. Para cada janela de um segundo, as métricas de entropia, variação de IPs origem e taxa de pacotes são calculadas e baseado nesses três valores  calcula-se a correlação (NaHiD) que comparando com as métricas de um tráfego normal, retorna uma resposta entre 0 e 1, que denota o quanto o tráfego analisado é parecido com um normal. Vários valores de limiar são testados para avaliarmos o comportamento do \textit{framework} em termos de taxa de acerto, falsos positivos e negativos.

\section{Contribuições}

A partir da implementação do \textit{framework}, foram avaliadas duas bases de dados DataMining e DARPA, as quais continham tráfegos maliciosos. Os gráficos mostraram um comportamento esperado para ambas as bases de dados, sendo que para cada valor de limiar escolhido, o comportamento da curva de taxas de acerto no primeiro \textit{dataset} possui valores próximos a 100\% para limiares de até 0.82, e no caso da base de dados DARPA, a detecção é máxima para uma larga faixa de limiar (entre 0.7 e 0.94). Ao comparar com \cite{HOQUE201748}, observou-se um comportamento muito semelhante para a base de dados DARPA. Conclui-se portanto que o sistema implementado é capaz de detectar ataques DDoS, sendo portanto, válido para detecção desse tipo de ataques, mostrando comportamento excelente ao ser avaliado segundo taxa de acertos. 

Como uma última contribuição deste trabalho, todos os códigos-fonte do \textit{framework} implementado, bem como os resultados e o código-fonte em \LaTeX \hspace{0.01cm} desse documento monográfico, estão disponíveis para acesso sob a licença GPL(\textit{Gnu Public License}) versão 3 no link : \url{https://github.com/jardelufc/RTDDoS}. Adicionalmente, acrescentamos que o código pode ser portado para rodar em Octave, visto que não foi utilizado nenhum \textit{toolbox} proprietário.   
\section{Trabalhos Futuros}
Como perspectivas futuras para o trabalho, as seguintes abordagens poderão ser adotadas:
\begin{itemize}
	\item Estudar algumas técnicas de \textit{machine learning} para aprimorar a detecção.
	\item Desenvolver o \textit{framework} com limiares adaptativos para maximizar os acertos em outras bases de dados e em situações reais.
	\item Comparar o \textit{framework} implementado com outros na literatura.
	\item Estudar modificações na correlação utilizada.
	\item Criar um ambiente controlado, com alguns \textit{hosts} e um servidor para simular ataques reais e testar o \textit{framework}.
	\item Criar uma aplicação a ser instalada no servidor, a qual ficaria analisando os pacotes e detectando possíveis ataques em quaisquer nós da rede.
\end{itemize}
