\chapter[Conclusões e Trabalhos Futuros]{Conclusões e Trabalhos Futuros}
% ----------------------------------------------------------

Este trabalho apresentou um estudo de quatro algoritmos de alocação de recursos de rádio para múltiplos serviços em redes celulares LTE de 4G baseadas em OFDMA, em que os quatro algoritmos foram escolhidos após uma extensão revisão bibliográfica. O estudo destes algoritmos se deu por meio da implementação de tais algoritmos para análise de desempenho com base na métrica de satisfação do usuário. O ambiente de simulação modelado e utilizado neste trabalho era composto por dois tipos de serviço: serviço CBR, o qual pode ser classificado como NRT, e o serviço VoIP, que é um serviço RT. 

A avaliação do desempenho dos quatro algoritmos estudados se deu por meio de simulações em ambientes com serviço único RT, serviço único NRT e múltiplos serviços. A partir dos resultados destas simulações, foram obtidas as cargas de usuários tal que o nível de satisfação dos usuários estava acima de 90\%. Esses valores foram então utilizados para a construção do plano de capacidade, que nos permite condensar o desempenho de todos os algoritmos em todos os cenários em um único gráfico. Note que com isso, todos os objetivos inicialmente estabelecidos para este trabalho foram atingidos. 

A partir da análise do desempenho dos algoritmos em todos estes cenários, conclui-se que o algoritmo que obteve o melhor foi o EXP/PF, já que ele obteve os melhores resultados para todos os cenários com serviço único e múltiplos serviços. O motivo para isso é que este algoritmo faz a alocação dos recursos para os serviços RT e NRT com base na principal métrica de QoS que define se os usuários destes serviços estão satisfeitos. O segundo melhor desempenho foi obtido pelo algoritmo \textit{Utility} Lei, o qual obteve o segundo melhor desempenho em todos os cenários analisados. A principal desvantagem do \textit{Utility} Lei é que ele aloca os recursos para as duas classes de serviço com base na métrica de atraso do pacote HOL, em vez de focar na alocação baseada em vazão de dados para o serviço NRT. Os dois piores desempenhos foram obtidos pelos algoritmos MDU e QHMLWDF. Em relação ao MDU, o principal motivo para o baixo desempenho é a maior prioridade absoluta dada para o serviço RT, o que deteriora a satisfação do serviço NRT. O QHMLWDF atingiu tão baixo desempenho pelo fato de utilizar o tamanho da fila absoluto para alocar os recursos para os dois serviços, o que prejudicou o seu desempenho pelo fato de os serviços analisados possuírem taxas de geração de pacotes bastantes distintas.   

Conclui-se, portanto, que os algoritmos de RRA são mecanismos que já foram e continuam sendo bastante estudados por pesquisadores da área das telecomunicações, como percebido pelo que foi apresentado no segundo capítulo. Pode-se concluir também que há diversas formas para formular/conceber algoritmos de RRA, desde abordagens heurísticas/experimentais até abordagens com uma sólida formulação matemática baseada em alguma teoria, tal como a teoria da utilidade. 

Segundo \citeonline{kim2009qos}, uma característica desejada que poderia estar presente em algoritmos de RRA é a flexibilidade para que as operadoras das redes celulares ajustem o ponto de operação dos algoritmos de acordo com decisões estratégicas ou segundo as mudanças de tendências dos clientes. Um ajuste de operação que poderia ser tomado de acordo com decisões estratégicas é dar maior prioridade para o serviço que gera maior quantidade de tráfego, o que resultaria em um aumento da vazão de dados total do sistema. Em relação as mudanças na tendência dos clientes, uma decisão que poderia ser tomada é atribuir maior prioridade para um determinado serviço que está sendo utilizado pela maioria dos usuários presentes no sistema. Nenhum dos algoritmos estudados aqui e nenhum outro algoritmo que foi encontrado na literatura se preocupa com este tipo de característica, o que abre espaço para possíveis trabalhos futuros.

Por fim, devido a essa grande área de pesquisa que é o estudo de algoritmos de RRA, existe a perspectiva que em trabalhos futuros do autor um algoritmo de RRA seja desenvolvido para ser aplicado em redes celulares 4G e 5ª Geração (5G) com múltiplos serviços. Pode-se adotar uma abordagem que possua uma fundamentação matemática ou mesmo uma abordagem heurística que seja eficaz. Como nenhum dos algoritmos apresentados aqui se preocupou com o que foi exposto em \citeonline{kim2009qos}, o algoritmo de RRA a ser desenvolvido deveria atender a este critério. Uma alternativa para isso é permitir que as operadoras de redes celulares tivessem a possibilidade de escolher um serviço para ter maior prioridade sobre o outro, em que o serviço protegido poderia permanecer com o nível de satisfação do usuário sempre acima de um limiar pré-determinado. Tal característica poderia ser alcança utilizando algum mecanismo de controle de satisfação feito de forma dinâmica, ou seja, em tempo de execução. Pode-se também investigar outras classes de serviços que tornem o cenário mais desafiador para os algoritmos de RRA, além de analisar o desempenho dos algoritmos em ambientes com múltiplas células, o que torna o cenário ainda mais desafiador devido a presença da interferência gerada por uma célula em células vizinhas. Para finalizar, devido a grande quantidade de usuários que está projetada para acessar as redes celulares 5G e a grande quantidade de dados que usuários vão demandar \cite{osseiran2014scenarios}, vale ressaltar que para que o algoritmo a ser desenvolvido possa ser aplicado em redes celulares 5G, é necessário que ele seja capaz de gerenciar de forma bastante eficiente os recursos de rádio para que os requerimentos de QoS dos usuários conectados sejam atendidos.