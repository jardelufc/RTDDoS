\chapter[Introdução]{Introdução}
\label{introducao}
% ---------------------------------------------------------- 

No início da década de noventa, a \textit{internet} era uma ferramenta disponível apenas em universidades, pra uso por professores, pesquisadores e alunos, os quais utilizavam computadores caros, pesados e conectados por um grande número de fios. No entanto, nos últimos trinta anos, devido a avanços tecnológicos de fabricação de semicondutores, houve uma miniaturização do computador, de maneira a caber na palma de nossas mãos, além de uma redução drástica do custo, e da facilidade de uso com a tecnologia \textit{touch screen} e de muitos outros avanços. 

Em paralelo com as melhorias dos dispositivos, também foi ampliada a abrangência e velocidade de comunicação de dados na \textit{internet}, principalmente pela evolução das tecnologias de comunicação sem fio, tais como ADSL (\textit{Asymmetric Digital Subscriber Line}), Wi-Fi e 4G/5G. Essa evolução concomitante de dispositivos e de tecnologias de suporte à comunicação fez a \textit{internet} deixar de ser uma ferramenta restrita a universidades para atingir todas as pessoas, de todas as idades, e de todas as classes sociais, que hoje fazem o uso da \textit{internet} em seu cotidiano para atividades, tais como, comunicar-se com amigos e familiares por mensagens instantâneas, chamadas de voz e fazer uso de aplicações \textit{online}, o que tem sido cada vez mais comum no contexto atual na rede mundial de computadores. Além disso, o número de serviços e comodidades fornecidos para os usuários apenas aumentam seja por aplicações \textit{web}, aplicativos de celular, ou até mesmo dispositivos IoT (\textit{Internet of Things}). Com essa difusão dessas tecnologias, atividades maliciosas na rede vêm surgindo em proporção cada vez maior, colocando em risco os serviços, bem como a integridade dos dados dos usuários. 

De acordo com \citeonline{barford2002signal}, redes que não possuem nenhum mecanismo de análise de tráfego, não podem garantir segurança aos clientes que a utilizam, pois não têm garantia de que podem operar eficientemente. Assim, redes que não possuem sistemas de detecção de atividades maliciosas estão sujeitos a terem suas funcionalidades comprometidas, ou mesmo invadidas por ataques produzidos por um agente malicioso na rede.

Um ataque consiste de uma atividade maliciosa que explora uma vulnerabilidade na rede ou em um nó da rede. Dentre os ataques em redes de computadores, o DoS (\textit{Denial of Service}) é o que ocasiona uma maior perturbação da qualidade de serviço da rede \cite{badishi2006exposing}. Um ataque DoS tenta tornar um nó na rede (geralmente servidor \textit{web}) indisponível, inundando-o com falsas requisições, ocupando a largura de banda e/ou consumindo seus recursos computacionais \cite{1638123}. Uma evolução deste tipo de ataque trata-se do DDoS, possuindo as mesmas características do anterior, porém sendo executado de forma distribuída, ou seja, mais de um atacante ou mesmo um atacante controlando remotamente outras vítimas, com o objetivo de tornar indisponível um servidor/serviço alvo. O primeiro registro de atividades DDoS aconteceu em Julho de 1999, na universidade de Minnesota - Estados Unidos, onde a rede da vítima ficou indisponível por mais que dois dias \cite{srivastava2011recent}. No ano seguinte, um ataque DDoS tornou-se conhecido mundialmente, quando um grande número de sites como o \textit{Yahoo, EBay, Amazon e CNN} ficaram inoperantes devido a ataques dessa natureza \cite{calce2008mafiaboy}.

Assim, sistemas de detecção de intrusão (IDS) são necessários para monitorar a rede em busca de pacotes maliciosos e assim identificá-los para tomar as medidas corretivas que impeçam o avanço do atacante.

O presente trabalho faz o estudo, implementação e validação de um \textit{framework} de detecção de ataques DDoS, o qual pode ser encontrado em \cite{HOQUE201748}. Tal arcabouço é capaz de detectar um tráfego malicioso utilizando uma correlação proposta pelos autores que considera poucos parâmetros de tráfego durante a análise, utilizando uma janela de tráfego de 1 segundo. Assim, neste trabalho é realizada a implementação em MATLAB para o \textit{framework} e a validação é mostrada a partir de duas bases de dados estudadas e testadas no \textit{framework}.  

\section{Objetivos}
O presente trabalho tem por objetivos gerais estudar, implementar e validar um \textit{framework} de detecção de ataques DDoS encontrado na literatura. No decorrer do desenvolvimento deste trabalho, os seguintes objetivos específicos foram perseguidos e atingidos:
\begin{itemize}
	\item Entender o funcionamento do \textit{framework} estudado.
	\item Implementar em MATLAB o \textit{framework} estudado.
	\item Simular o \textit{framework} bases de dados para validar os resultados com o artigo de origem.
\end{itemize}
\section{Organização da monografia}
Os estudos deste trabalho estão organizados da seguinte forma: No \capref{fundamentacao} é apresentado um estudo bibliográfico sobre ameaças de rede, ataques DDoS e sistemas IDS (\textit{Intrusion Detection System}). No \capref{metodologia}, mostramos a modelagem do ambiente de simulação utilizado neste trabalho é descrita. No \capref{resultados}, apresentamos o desempenho obtido pelo \textit{framework} estudado por meio da taxa de acerto para cada janela de tráfego. Por fim, o último capítulo deste trabalho apresenta as conclusões realizadas a partir dos resultados obtidos, além de algumas perspectivas para a continuação deste trabalho. 