\chapter[Introdução]{Introdução}
\label{introducao}
% ----------------------------------------------------------
Ataques Distribuídos de Negação de Serviço (do inglês, DDoS) são uma ameaça a servidores de redes online, tais como servidores de sites web e servidores em nuvem. O objetivo desse tipo de ataque intencional é inundar o alvo com requisições e assim deixá-lo indisponível na rede. Existem essencialmente três tipos de ataques: Negação distribuída, Handshake e UDP. O primeiro caracteriza-se por requisições abertas por um grande número de computadores infectados. No segundo, faz-se uma comunicação inicial com o alvo que não é completada, mantendo assim o servidor esperando indefinidamente. Já no terceiro, fluxos falsos UDP são criados com o mesmo objetivo de tornar o serviço inoperante. 
Os métodos estatísticos existentes na literatura para análise de ataque DDoS falham principalmente devido às correlações de deslocamento, escala e deslocamento-escala ao longo de tráfegos de rede, gerando assim uma grande ocorrência de falsos positivos. Além disso, métodos estatísticos impõem alto overhead computacional quando um grande número de objetos é incluído para análise. Consequentemente, tais métodos falham em realizar detecção de ataque DDoS em tempo real. Algumas medidas de correlação tais como Pearson, Spearman e Kendall falham em identificar a diferença entre um pacote normal e um malicioso quando há valores correlacionados entre os pacotes.
De fato, um método de detecção de ataques DDoS precisa considerar poucos parâmetros de tráfego durante a análise, tal como o método chamado NaHiDVERC \cite{HOQUE201748}, o qual analisa apenas entropia de IPs e taxa de pacote. Tendo em vista uma implementação em software e hardware, este método será utilizado em nosso trabalho, visto que é facilmente implementável em hardware. O método computa dois valores: a distância absoluta e o desvio entre A e B a partir da média e do desvio padrão. Se a entropia de IPs origem em um pequeno intervalo de tempo é alta e a taxa de pacote é também muito alta, a probabilidade de ataque é alta. Se a variação entre IPs origem é muito alta e a taxa de pacote também é alta, a probabilidade de ataque é alta. O framework tem como objetivo detectar ataques DDoS em tempo real no computador alvo. Trata-se de uma combinação entre aplicações em software e hardware, para classificar um tráfego como normal ou ataque com uma taxa aceitável de acertos. Tal arcabouço possui três componentes: pré-processamento, um módulo de hardware dedicado para detecção e um gerente de segurança. Neste trabalho os componentes um e três serão trabalhados. Além disso, é necessária a presença de um roteador para capturar tráfego e duas bases de dados. Amostras de tráfego serão capturadas de uma porta do roteador como um pacote TCP/IP, que são enviadas ao módulo de pré-processamento. Nessa fase, a cada segundo, os pacotes recebidos são agrupados e essa instância de tráfego é enviada para o módulo de detecção de ataque, que irá classificar a instância como normal ou ataque. O gerente de segurança manterá um perfil normal e um valor limiar em sua base de perfis, para ser usado pelo módulo de detecção. Incrementalmente, o gerente recalcula o perfil normal e o limiar baseado nos valores anteriores. 
Existem duas abordagens durante a análise do tráfego: uma considerando apenas a informação no cabeçalho do pacote ou se o cabeçalho e dados estarão juntos. Nas duas formas os campos dos pacotes são analisados para detectar alguma anomalia na rede. IP e porta origem/destino, protocolos e flags do cabeçalho TCP são úteis para detectar pacotes maliciosos. Assim, a entropia e a variação entre IPs origem e taxa de pacotes são calculados para cada amostra de tráfego.
O último módulo, que é o módulo de segurança irá operar offline e fará análises detalhadas dos logs de detecção usando técnicas de machine learning e estatística. Além disso, feedbacks de especialistas podem ser utilizados para validar os resultados. Inicialmente, o gerente vai calcular um perfil de tráfego normal que melhor representa instâncias desse tráfego para treinamento. Esses valores serão carregados na base de dados. Vale ressaltar que esses valores serão modificados dinamicamente de acordo com as amostras de tráfego. 


\section{Organização da monografia}
Os estudos deste trabalho estão organizados da seguinte forma: No próximo capítulo será apresentado um estudo bibliográfico sobre ameaças de rede e ataques DDoS. No \capref{metodologia}, a modelagem do ambiente de simulação utilizado neste trabalho é descrita. No \capref{resultados}, apresentamos o desempenho obtido pelo \textit{framework} estudado por meio da taxa de acerto para cada janela de tráfego. Por fim, o último capítulo deste trabalho apresenta as conclusões realizadas a partir dos resultados obtidos e algumas perspectivas para a continuação deste trabalho. 