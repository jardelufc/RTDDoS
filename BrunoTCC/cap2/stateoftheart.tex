\chapter[Revisão Bibliográfica]{Revisão Bibliográfica}
% ----------------------------------------------------------
Um dos objetivos desse trabalho é o estudo bibliográfico sobre algoritmos de alocação de recursos de rádio para sistemas celulares de 4ª geração. Nesta seção, serão apresentados alguns trabalhos encontrados na literatura que propuseram algoritmos de RRA para cenários com serviço único RT, com serviço único NRT ou cenários com múltiplos serviços.    

\section{Algoritmos para serviço único RT}

Algoritmos para cenários com serviço único RT têm sido bastante analisados na literatura. Visando utilizar os recursos do sistema da forma mais eficiente possível e atribuindo maior prioridade para usuários RT com piores níveis de QoS (atrasos maiores), alguns algoritmos oportunísticos de escalonamento de pacotes utilizaram funções de prioridade que usam a taxa de transmissão instantânea como indicador de eficiência e um indicador de QoS baseado em atraso \cite{ho2011opportunistic,Braga2006,Gueguen2008}. Em \citeonline{shakkottai2002scheduling}, outro algoritmo de RRA conhecido como \ac{EXP}, que foi inicialmente concebido para sistemas \ac{CDMA} com uma única portadora e com o espectro do enlace direto compartilhado, foi estudado, o qual utiliza o atraso dos pacotes para alocar os recursos para os usuários.

Um algoritmo heurístico de escalonamento de pacotes com o objetivo principal de maximizar a satisfação dos usuários RT presentes no sistema foi proposto por \citeonline{Lima2009}, e foi estudado com maiores detalhes para o cenário com múltiplos serviços em \citeonline{Lima2010}. Até onde sabemos, estes foram os primeiros trabalhos que abordaram o problema específico de maximizar a satisfação dos usuários em cenários com serviço único RT. 

Algoritmos baseados em teoria da utilidade (a qual será explicada com mais detalhes no capítulo \ref{benchmarking}) que lidam com serviço único RT foram estudados em  \citeonline{Art:Song2005_Commag}, em que os autores utilizaram algumas funções de utilidade que foram especialmente concebidas para os cenários estudados por eles. \citeonline{lei2007qos} propôs um algoritmo$^{1}$ \footnotetext[1]{O autor refere-se ao algoritmo apenas como \textit{Utility} por ser o único algoritmo analisado por ele que utiliza a teoria da utilidade. Como analisaremos outros algoritmos baseados nessa teoria, iremos nos referir a esse algoritmo como \textbf{Utility Lei} (nome do primeiro autor do artigo).}, também baseado em teoria da utilidade, que tem como objetivo garantir os requerimentos de Taxa de Perda de Pacote (PLR, do inglês \textit{Packet Loss Ratio}) e taxa de interrupção de reprodução de serviços de vídeo, em que uma função barreira foi utilizada como a função de utilidade. Um algoritmo conhecido como \ac{UEPS}, que também é baseado em teoria da utilidade, foi proposto por \citeonline{Proc:Ryu2005}, em que uma função de utilidade baseada no atraso dos pacotes é utilizada como fator de urgência e o estado do canal do usuário é utilizado para indicar a eficiência do uso dos recursos de rádio. \citeonline{Rodrigues2014_Wiley} propôs um algoritmo de RRA, chamado de \ac{TSM}, com um arcabouço unificado baseado em teoria da utilidade que utiliza uma função de utilidade com formato de S para a maximização da satisfação dos usuários em cenários com serviço único RT.

\section{Algoritmos para serviço único NRT}

Um outro campo de estudo é o de algoritmos de RRA para cenários com serviço único NRT, o qual também tem sido bastante estudado, como por exemplo em \citeonline{tarchi2010analysis} e \citeonline{pitic2010performance}.

\citeonline{Jang2003} analisou o algoritmo \ac{RM}, o qual tem o objetivo de maximizar a capacidade do sistema por meio da estratégia de alocar cada \acs{RB} para usuários que puderem transmitir com a maior taxa naquele RB, caracterizando um abordagem heurística. O algoritmo \ac{PF} tem o objetivo de alocar os recursos de rádio para usuários com boas condições de canal; no entanto, esse algoritmo não pode ser diretamente utilizado para cenários com serviço único RT, pois é menos eficiente em ambientes com aplicações sensíveis ao atraso \cite{kwan2009proportional, song2004joint,Proc:Ryu2005}. Nesse tipo de cenário, o algoritmo PF não consegue garantir que os requerimentos de QoS de serviços RT sejam atendidos devido a pequena quantidade de tráfego que esses serviços geram \cite{Proc:Lei2007}. Outra explicação é que algoritmos que alocam recursos baseados apenas na condição de canal dos usuários possuem uma pequena região de estabilidade, não sendo assim adequados para ambientes com serviço único RT \cite{Art:Song2005_Commag}. Alguns estudos encontrados na literatura mostram que o PF realmente não possui bom desempenho em cenários com serviço sensíveis ao atraso, tais como VoIP e vídeo \cite{Ali2012, basukala2009performance, capozzi2013downlink}.
Esse dois algoritmos, RM e PF, são considerados algoritmos clássicos para alocação de recursos para cenários com serviço único RT. 

O algoritmo \ac{SORA-NRT} foi inicialmente proposto por \citeonline{Santos2007} e foi estudando com mais detalhes em \citeonline{Lima2010} para o cenário com múltiplos serviços. 
\citeonline{rodrigues2009adaptive} propuseram um arcabouço adaptativo que pode ser configurado com estratégias diferentes de acordo com os objetivos do operador do sistema; esse arcabouço explora os critérios utilizados pelos algoritmos \ac{RM}, \ac{PF} e \ac{MMF} para fornecer uma alocação de recursos justa e maximizar a satisfação dos usuários. Até onde sabemos, o problema específico de maximização da satisfação dos usuários em cenários com serviço único NRT foi primeiramente estudando por \citeonline{Santos2007}, \citeonline{rodrigues2009adaptive} e \citeonline{Lima2010}. O estudo deste problema foi aprofundado ainda mais por \citeonline{Rodrigues2014_Wiley}, onde um arcabouço baseado em teoria da utilidade foi proposto, o \ac{DSM}, o qual utiliza uma função de utilidade em formato de S para atingir altos níveis de satisfação do usuário.

\section{Algoritmos para múltiplos serviços}

Além dos cenários com serviço único, o cenário com múltiplos serviços, que é mais realista e complexo, também foi consideravelmente analisado na literatura. 

Um cenário composto por apenas um usuário CBR e um usuário VBR (do inglês \textit{Variable Bit Rate}) foi estudado por \citeonline{Hoo1998}. Algoritmos de RRA subótimos foram propostos para o caso com múltiplos usuários e serviços por \citeonline{Wang2004} e \citeonline{Yu2007}. Um algoritmo de alocação de recursos baseado em teoria dos jogos que utiliza funções de utilidade e um escalonador de pacotes baseado em atraso foi estudo por \citeonline{Ali2012}.

\citeonline{wu2015qos} propuseram um algoritmo de RRA que tem o objetivo de minimizar a PLR de serviços RT e, ao mesmo tempo, garantir os requerimentos de QoS dos outros serviços presentes no sistema; neste trabalho, os autores apresentaram alguns resultados mostrando que o algoritmo proposto obteve um bom compromisso entre satisfação dos requerimentos de QoS e justiça na alocação de recursos. \citeonline{kim2009qos} utilizaram Algoritmos de Mapeamento Bipartite (do inglês \textit{Bipartite Matching Algorithm}) para satisfazer os requerimentos de QoS de serviços RT e um algoritmo de gradiente padrão para maximizar a utilidade de serviços NRT. 

O algoritmo clássico de RRA conhecido como \ac{MLWDF} considera o estado do canal do usuário e tamanho da fila com o objetivo de manter o atraso do pacote do Começo da Fila (HOL, do inglês \textit{Head-of-Line}) da maioria dos usuários abaixo de um determinado requerimento \cite{Art:Andrews2001}; este algoritmo pode ser utilizado em cenários de múltiplos serviços, em que a definição do serviço com maior prioridade é feita de acordo com a tolerância do atraso do pacote HOL. O algoritmo \ac{MLWDF} foi estendido por \citeonline{Nasralla2013}, onde foi chamado de \ac{QHMLWDF}, para que a alocação dos recursos fosse feita levando em consideração o tamanho da fila (em bits) e o atraso do pacote HOL.

\citeonline{Phd:Song2005} propôs um algoritmo baseado em teoria da utilidade (o qual foi inicialmente apresentado em \citeonline{song2004joint}), chamado de \ac{MDU}, que utiliza funções de utilidade do tipo degrau para definir a prioridade dos usuários na alocação dos recursos. Um algoritmo de escalonamento de pacotes baseado na Informação do Estado do Canal (CSI, do inglês \textit{Channel State Information}) e funções de utilidade baseadas no atraso do pacote HOL foi proposto por \citeonline{Proc:Lei2007}, em que uma função com formato de Z é utilizada para escalonar os usuários RT e uma função exponencial é usada para alocar recursos para usuários NRT.

\citeonline{basukala2009performance} estenderam o algoritmo clássico \ac{EXP} e propuseram um algoritmo de RRA, chamado de \textit{Exponential}/PF (EXP/PF), que escalona os usuários RT de acordo com a política do algoritmo \ac{EXP} e usuários NRT segunFdo o algoritmo \ac{PF}. Um outro algoritmo de RRA foi proposto por \citeonline{indumathi2011user} em que a prioridade dos pacotes durante a alocação dos recursos é definida de acordo com o atraso, tamanho do pacote e nível de prioridade de QoS dos pacotes. 

Os algoritmos para serviço único propostos por \citeonline{Lima2009} e \citeonline{Santos2007} foram ampliados para cenários com múltiplos serviços por \citeonline{Lima2010}, onde o algoritmo \textit{Capacity-driven Resource Allocation} (CRA) foi proposto, o qual controla dinamicamente o compartilhamento dos recursos entre os serviços (RT e NRT) e explora o conhecimento da qualidade de canal para proporcionar ganhos na capacidade conjunta do sistema.