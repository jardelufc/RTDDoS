\chapter[Revisão Bibliográfica]{Revisão Bibliográfica}
% ----------------------------------------------------------
Ataques distribuídos de negação de serviço (do inglês, DDoS) figuram uma das principais ameaças de segurança atualmente. Organizações perdem financeiramente e principalmente tendem a ter seu valor questionado quando são vítimas de invasores na rede.

Um ataque de negação de serviço(DoS - Denial-of-Service) torna um componente de rede inutilizável por usuários que estejam consumindo o serviço fornecido. A maioria dos ataques DoS na internet pode ser dividida em três categorias: \cite{kurose}
\begin{itemize}
	\item Ataque de vulnerabilidade: Mensagens são enviadas a uma aplicação vulnerável ou a um servidor, sendo executado em um hospedeiro alvo.
	\item Inundação na largura de banda: O atacante envia um grande número de pacotes maliciosos ao hospedeiro alvo até que o enlace de acesso do alvo se entope, impedindo os pacotes legítimos de alcançarem o servidor. 
	\item Inundação na conexão: O atacante estabelece um grande número de conexões TCP semiabertas ou abertas no hospedeiro alvo. 
\end{itemize}
Já segundo \cite{ddosatacks}, os ataques DoS são classificados em 5 categorias baseadas no protocolo cujo é atacado: Dispositivo, sistema operacional, aplicação, inundação de dados e características do protocolo. O primeiro inclui ataques que podem ser causados ao tirar vantagem de \textit{bugs} ou vulnerabilidades em software. O segundo leva em consideração ataques que aproveitam-se das formas nas quais os sistemas operacionais implementam os protocolos. Ataques baseados na aplicação infectam o alvo por meio de \textit{bugs} específicos da rede e tentam drenar os recursos da vítima. Em ataques baseados em inundação de dados, um atacante tenta usar a largura de banda disponível para mandar quantidades massivas de dados, fazendo com que o alvo processe essa grande quantidade. Por fim, ataques baseados em características do protocolo são caracterizados por tirarem vantagens de certos padrões de protocolo.
 
Um ataque distribuído de negação de serviço  

Falar sobre ataques, definir objeto de tráfego, sniffer IDS vulnerabilidades...
   

