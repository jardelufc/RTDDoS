\documentclass[
	% -- opções do pacote babel --
	english,		% idioma adicional para hifenização
	portuguese,			% o último idioma é o principal do documento
	% -- opções da classe memoir --
	12pt,				% tamanho da fonte
	openright,			% capítulos começam em pág ímpar (insere página vazia caso preciso)
	oneside,			% para impressão em verso e anverso. Oposto a oneside
	a4paper,			% tamanho do papel. 
	% -- opções da classe abntex2 --
	%chapter=TITLE,		% títulos de capítulos convertidos em letras maiúsculas
	%section=TITLE,		% títulos de seções convertidos em letras maiúsculas
	%subsection=TITLE,	% títulos de subseções convertidos em letras maiúsculas
	%subsubsection=TITLE,% títulos de subsubseções convertidos em letras maiúsculas
	% -- opções de sumário -- %
	sumario=abnt-6027-2012 % sumário conforme as recomendações da ABNT NBR 6027:2012 (padrão)
%	sumario=tradicional, % usa o estilo tradicional de sumários do memoir
	%	.	
	]{abntex2}

% Modificacoes pessoais de acordo com a norma da UFC (?)
\renewcommand{\ABNTEXchapterfontsize}{\bfseries\normalsize}
\renewcommand{\ABNTEXchapterfont}{\normalfont}
\setboolean{ABNTEXupperchapter}{true}
\renewcommand{\ABNTEXsectionfontsize}{\bfseries\normalsize}
\renewcommand{\ABNTEXsubsectionfontsize}{\bfseries\itshape\normalsize}
\renewcommand{\ABNTEXsubsubsectionfontsize}{\itshape\normalsize}
\renewcommand{\ABNTEXsubsubsubsectionfontsize}{\normalsize}

\renewcommand{\cftfigurename}{Figura\space}
\renewcommand{\cfttablename}{Tabela\space}

\setlength{\cftbeforechapterskip}{0em}

% --- 
% CONFIGURAÇÕES DE PACOTES
% --- 

%%%%%%%%%%%%%%%%%%%%%%%%%%%%%%%%%%%%%%%%%%%%%%%%%%%%%%%%%%%%%%%%
%%% I had to include this package here because there was an
%%% incompatibility with another package
\usepackage[svgnames]{xcolor}
%%%%%%%%%%%%%%%%%%%%%%%%%%%%%%%%%%%%%%%%%%%%%%%%%%%%%%%%%%%%%%%%

\usepackage{pdfpages}

\usepackage{pbox}
% ---
% Pacotes básicos 
% ---

\usepackage{babel}
\usepackage{lmodern}			% Usa a fonte Latin Modern			
\usepackage[T1]{fontenc}		% Selecao de codigos de fonte.
\usepackage[utf8]{inputenc}		% Codificacao do documento (conversão automática dos acentos)
\usepackage{lastpage}			% Usado pela Ficha catalográfica
\usepackage{indentfirst}		% Indenta o primeiro parágrafo de cada seção.
\usepackage{color}				% Controle das cores
\newcommand{\ownership}{\flushleft\footnotesize Fonte: Elaborada pelo autor.}
\newcommand{\source}{\flushleft\footnotesize Fonte: }
\usepackage{microtype} 			% para melhorias de justificação
\usepackage{ufc-abntex2}

%%%%%%%%%%%%%%%%%%%%%%%%%%%%%%%%%%%%%%%%%%%%%%%%%%%%%%%%%%%%%%%%
%%% Para incluir gráficos -- Formato .eps
\graphicspath{{.}{./Figures/}}
\DeclareGraphicsExtensions{.pdf,.eps}
%%%%%%%%%%%%%%%%%%%%%%%%%%%%%%%%%%%%%%%%%%%%%%%%%%%%%%%%%%%%%%%%

%%%%%%%%%%%%%%%%%%%%%%%%%%%%%%%%%%%%%%%%%%%%%%%%%%%%%%%%%%%%
%%% Incluir pacote para gerenciar espaço entre valores e unidades
\usepackage{siunitx}
\sisetup{%
	binary-units,
	per-mode = symbol,
	detect-all,
	list-units = single,
	list-final-separator = { and },
	%range-phrase={--},
	range-units = single,
	round-mode = places,
	round-precision = 2,
	product-units = single,
}
%%%%%%%%%%%%%%%%%%%%%%%%%%%%%%%%%%%%%%%%%%%%%%%%%%%%%%%%%%%%

% Maths
\usepackage{amsthm}
\usepackage{amsmath,dsfont,xfrac,bm}            
% Organizar numeração das figuras por capítulo 
\numberwithin{figure}{chapter}
% Organizar numeração das tabelas por capítulo
\numberwithin{table}{chapter}

\usepackage{subcaption}

\usepackage{multirow}

%%%%%%%%%%%%%%%%%%%%%%%%%%%%%%%%%%%%%%%%%%%%%%%%%%%%%%%%%%%%
% Package algorithm -- pacote para manipulação de pseudo-código.
\usepackage{algorithm}
\usepackage{algorithmicx}
\usepackage{algpseudocode}

%%%%%%%%%%%%%%%%%%%%%%%%%%%%%%%%%%%%%%%%%%%%%%%%%%%%%%%%%%%%
%%% Package enumerare -- Extensão do ambiente "enumerate".
\usepackage{enumerate}
		
% ---
% Pacotes adicionais, usados apenas no âmbito do Modelo Canônico do abnteX2
% ---

% Acronyms
%\usepackage[nomain,acronym,toc,shortcuts]{glossaries}
\usepackage[nogroupskip,acronym,shortcuts]{glossaries}
\newglossary{symbols}{sym}{sbl}{\symbname} % glossary for symbols
\makenoidxglossaries

% Style witlonder length for description https://tex.stackexchange.com/questions/21666/glossary-term-list-width
\newglossarystyle{clong}{%
	\renewenvironment{theglossary}%
	{\begin{longtable}{lp{1.5\glsdescwidth}}}%
		{\end{longtable}}%
	\renewcommand*{\glossaryheader}{}%
	\renewcommand*{\glsgroupheading}[1]{}%
	\renewcommand*{\glossaryentryfield}[5]{%
		\glstarget{##1}{##2} & ##3\glspostdescription\space ##5\\}%
	\renewcommand*{\glossarysubentryfield}[6]{%
		& \glstarget{##2}{\strut}##4\glspostdescription\space ##6\\}%
	%\renewcommand*{\glsgroupskip}{ & \\}%
}
\newacronym{3GPP}{3GPP}{3rd Generation Partnership Project}
\newacronym{3GPP2}{3GPP2}{3rd Generation Partnership Project 2}
\newacronym{1G}{1G}{1$^\text{st}$~Generation}
\newacronym{2G}{2G}{2$^\text{nd}$~Generation}
\newacronym{3G}{3G}{3$^\text{rd}$~Generation}
\newacronym{4G}{4G}{4$^\text{th}$~Generation}
\newacronym{5G}{5G}{5$^\text{th}$~Generation}
\newacronym{AA}{AA}{Antenna Array}
\newacronym{AC}{AC}{Admission Control}
\newacronym{AD}{AD}{Attack-Decay}
\newacronym{ABS}{ABS}{Almost Blank Subframe}
\newacronym{ADSL}{ADSL}{Asymmetric Digital Subscriber Line}
\newacronym{AHW}{AHW}{Alternate Hop-and-Wait}
\newacronym{AMC}{AMC}{Adaptive Modulation and Coding}
\newacronym{AP}{AP}{Access Point}
\newacronym{APA}{APA}{Adaptive Power Allocation}
\newacronym{ARMA}{ARMA}{Autoregressive Moving Average}
\newacronym{ATES}{ATES}{Adaptive Throughput-based Efficiency-Satisfaction Trade-Off}
\newacronym{AWGN}{AWGN}{Additive White Gaussian Noise}
\newacronym{BB}{BB}{Branch and Bound}
\newacronym{BC}{BC}{Branch and Cut}
\newacronym{BD}{BD}{Block Diagonalization}
\newacronym{BER}{BER}{Bit Error Rate}
\newacronym{BF}{BF}{Best Fit}
\newacronym{BL}{BL}{Bit Loading}
\newacronym{BLER}{BLER}{BLock Error Rate}
\newacronym{BPC}{BPC}{Binary Power Control}
\newacronym{BPSK}{BPSK}{Binary Phase-Shift Keying}
\newacronym{BRA}{BRA}{Balanced Random Allocation}
\newacronym{BS}{BS}{Base Station}
\newacronym{CAP}{CAP}{Combinatorial Allocation Problem}
\newacronym{CAPEX}{CAPEX}{Capital Expenditure}
\newacronym{CBF}{CBF}{Coordinated Beamforming}
\newacronym{CBR}{CBR}{Constant Bit Rate}
\newacronym{CBS}{CBS}{Class Based Scheduling}
\newacronym{CC}{CC}{Congestion Control}
\newacronym{CDF}{CDF}{Cumulative Distribution Function}
\newacronym{CDMA}{CDMA}{Code-Division Multiple Access}
\newacronym{CL}{CL}{Closed Loop}
\newacronym{CLPC}{CLPC}{Closed Loop Power Control}
\newacronym{CNR}{CNR}{Channel-to-Noise Ratio}
\newacronym{CPA}{CPA}{Cellular Protection Algorithm}
\newacronym{CPICH}{CPICH}{Common Pilot Channel}
\newacronym{CoMP}{CoMP}{Coordinated Multi-Point}
\newacronym{CQI}{CQI}{Channel Quality Indicator}
\newacronym{CRA}{CRA}{Capacity-driven Resource Allocation}
\newacronym{CRE}{CRE}{Cell Range Expansion}
\newacronym{CRM}{CRM}{Constrained Rate Maximization}
\newacronym{CRN}{CRN}{Cognitive Radio Network}
\newacronym{CS}{CS}{Coordinated Scheduling}
\newacronym{CSI}{CSI}{Channel State Information}
\newacronym{CTS}{CTS}{Clear to Send}
\newacronym{CUE}{CUE}{Cellular User Equipment}
\newacronym{D2D}{D2D}{Device-to-Device}
\newacronym{DCA}{DCA}{Dynamic Channel Allocation}
\newacronym{DE}{DE}{Differential Evolution}
\newacronym{DF}{DF}{Decode and Forward}
\newacronym{DFT}{DFT}{Discrete Fourier Transform}
\newacronym{DIST}{DIST}{Distance}
\newacronym{DL}{DL}{Downlink}
\newacronym{DMA}{DMA}{Double Moving Average}
\newacronym{DMRS}{DMRS}{Demodulation Reference Signal}
\newacronym{D2DM}{D2DM}{D2D Mode}
\newacronym{DMS}{DMS}{D2D Mode Selection}
\newacronym{DPC}{DPC}{Dirty Paper Coding}
\newacronym{DRA}{DRA}{Dynamic Resource Assignment}
\newacronym{DSA}{DSA}{Dynamic Spectrum Access}
\newacronym{DSM}{DSM}{Delay-based Satisfaction Maximization}
\newacronym{ECC}{ECC}{Electronic Communications Committee}
\newacronym{EE}{EE}{Energy Efficiency}
\newacronym{EFLC}{EFLC}{Error Feedback Based Load Control}
\newacronym{EI}{EI}{Efficiency Indicator}
\newacronym{e-ICIC}{e-ICIC}{Enhanced Inter-Cell Interference Coordination}
\newacronym{eNB}{eNB}{Evolved Node B}
\newacronym{EPA}{EPA}{Equal Power Allocation}
\newacronym{EPC}{EPC}{Evolved Packet Core}
\newacronym{EPS}{EPS}{Evolved Packet System}
\newacronym{E-UTRAN}{E-UTRAN}{Evolved Universal Terrestrial Radio Access Network}
\newacronym{ES}{ES}{Exhaustive Search}
\newacronym{EXP}{EXP}{Exponential}
\newacronym{FCP}{FCP}{Fundamental Counting Principle}
\newacronym{FD}{FD}{Full-Duplex Communications}
\newacronym{FDD}{FDD}{Frequency Division Duplex}
\newacronym{FDM}{FDM}{Frequency Division Multiplexing}
\newacronym{FDMA}{FDMA}{Frequency Division Multiple Access}
\newacronym{FER}{FER}{Frame Erasure Rate}
\newacronym{FF}{FF}{Fast Fading}
\newacronym{FSB}{FSB}{Fixed Switched Beamforming}
\newacronym{FST}{FST}{Fixed SNR Target}
\newacronym{FTP}{FTP}{File Transfer Protocol}
\newacronym{GA}{GA}{Genetic Algorithm}
\newacronym{GAP}{GAP}{Generalized Assignment Problem}
\newacronym{GAP-MQ}{GAP-MQ}{Generalized Assignment Problem with Minimum Quantities}
\newacronym{GBR}{GBR}{Guaranteed Bit Rate}
\newacronym{GLR}{GLR}{Gain to Leakage Ratio}
\newacronym{GOS}{GOS}{Generated Orthogonal Sequence}
\newacronym{GPL}{GPL}{GNU General Public License}
\newacronym{GRP}{GRP}{Grouping}
\newacronym{GTEL}{GTEL}{Wireless Telecommunications Research Group}
\newacronym{GSM}{GSM}{Global System for Mobile Communications}
\newacronym{HARQ}{HARQ}{Hybrid Automatic Repeat Request}
\newacronym{HetNet}{HetNet}{Heterogeneous Network}
\newacronym{HH}{HH}{Hughes-Hartogs}
\newacronym{HMS}{HMS}{Harmonic Mode Selection}
\newacronym{HOL}{HOL}{Head Of Line}
\newacronym{HSDPA}{HSDPA}{High-Speed Downlink Packet Access}
\newacronym{HSPA}{HSPA}{High Speed Packet Access}
\newacronym{HTTP}{HTTP}{HyperText Transfer Protocol}
\newacronym{ICMP}{ICMP}{Internet Control Message Protocol}
\newacronym{ICI}{ICI}{Intercell Interference}
\newacronym{ICIC}{ICIC}{Inter-Cell Interference Coordination}
\newacronym{ID}{ID}{Identification}
\newacronym{IETF}{IETF}{Internet Engineering Task Force}
\newacronym{ILP}{ILP}{Integer Linear Program}
\newacronym{UID}{UID}{Unique Identification}
\newacronym{IID}{IID}{Independent and Identically Distributed}
\newacronym{IIR}{IIR}{Infinite Impulse Response}
\newacronym{IMT}{IMT}{International Mobile Telecommunications}
\newacronym{INV}{INV}{Inverted Norm-based Grouping}
\newacronym{IoT}{IoT}{Internet of Things}
\newacronym{IP}{IP}{Internet Protocol}
\newacronym{IPv6}{IPv6}{Internet Protocol Version 6}
\newacronym{ISD}{ISD}{Inter-Site Distance}
\newacronym{ISI}{ISI}{Inter Symbol Interference}
\newacronym{ISM}{ISM}{Industrial, Scientific and Medical}
\newacronym{ITU}{ITU}{International Telecommunication Union}
\newacronym{JOAS}{JOAS}{Joint Opportunistic Assignment and Scheduling}
\newacronym{JOS}{JOS}{Joint Opportunistic Scheduling}
\newacronym{JP}{JP}{Joint Processing}
\newacronym{JRAPAP}{JRAPAP}{Joint RB Assignment and Power Allocation Problem}
\newacronym{JS}{JS}{Jump-Stay}
\newacronym{JSM}{JSM}{Joint Satisfaction Maximization}
\newacronym{KKT}{KKT}{Karush-Kuhn-Tucker}
\newacronym{LAC}{LAC}{Link Admission Control}
\newacronym{LA}{LA}{Link Adaptation}
\newacronym{LC}{LC}{Load Control}
\newacronym{LOS}{LOS}{Line of Sight}
\newacronym{LP}{LP}{Linear Programming}
\newacronym{LTE}{LTE}{Long Term Evolution}
\newacronym{LTE-A}{LTE-A}{\ac{LTE}-Advanced}
\newacronym{LTE-Advanced}{LTE-Advanced}{Long Term Evolution Advanced}
\newacronym{M2M}{M2M}{Machine-to-Machine}
\newacronym{MAC}{MAC}{Medium Access Control}
\newacronym{MANET}{MANET}{Mobile Ad hoc Network}
\newacronym{MC}{MC}{Modular Clock}
\newacronym{MCS}{MCS}{Modulation and Coding Scheme}
\newacronym{MDB}{MDB}{Measured Delay Based}
\newacronym{MDI}{MDI}{Minimum D2D Interference}
\newacronym{MDSM}{MDSM}{Modified Delay-based Satisfaction Maximization}
\newacronym{MDU}{MDU} {Max-Delay-Utility}
\newacronym{MF}{MF}{Matched Filter}
\newacronym{MG}{MG}{Maximum Gain}
\newacronym{MH}{MH}{Multi-Hop}
\newacronym{MILP}{MILP}{Mixed Integer Linear Problem} %{Mixed Integer Linear Programming}
\newacronym{MIMO}{MIMO}{Multiple Input Multiple Output}
\newacronym{MINLP}{MINLP}{Mixed Integer Nonlinear Programming}
\newacronym{MIP}{MIP}{Mixed Integer Programming}
\newacronym{MISO}{MISO}{Multiple Input Single Output}
\newacronym{MLWDF}{MLWDF}{Modified Largest Weighted Delay First}
\newacronym{MME}{MME}{Mobility Management Entity}
\newacronym{MMF}{MMF}{Max-Min Fairness}
\newacronym{MMSE}{MMSE}{Minimum Mean Square Error}
\newacronym{mmW}{mmW}{Millimeter Wave}
\newacronym{MOS}{MOS}{Mean Opinion Score}
\newacronym{MPF}{MPF}{Multicarrier Proportional Fair}
\newacronym{MPRP}{MPRP}{Maximization of the Product of the Residual Powers}
\newacronym{MRA}{MRA}{Maximum Rate Allocation}
\newacronym{MR}{MR}{Maximum Rate}
\newacronym{MRC}{MRC}{Maximum Ratio Combining}
\newacronym{MRT}{MRT}{Maximum Ratio Transmission}
\newacronym{MRUS}{MRUS}{Maximum Rate with User Satisfaction}
\newacronym{MS}{MS}{Mode Selection}
\newacronym{MSE}{MSE}{Mean Squared Error}
\newacronym{MSI}{MSI}{Multi-Stream Interference}
\newacronym{MTC}{MTC}{Machine-Type Communication}
\newacronym{MTSI}{MTSI}{Multimedia Telephony Services over IMS}
\newacronym{MTSM}{MTSM}{Modified Throughput-based Satisfaction Maximization}
\newacronym{MU-MIMO}{MU-MIMO}{Multi-User Multiple Input Multiple Output}
\newacronym{MU}{MU}{Multi-User}
\newacronym{Multi-CUT}{Multi-CUT}{Multi-Cell and Multi-User and Multi-Tier}
\newacronym{NAS}{NAS}{Non-Access Stratum}
\newacronym{NB}{NB}{Node B}
\newacronym{NCL}{NCL}{Neighbor Cell List}
\newacronym{NLP}{NLP}{Nonlinear Programming}
\newacronym{NLOS}{NLOS}{Non-Line of Sight}
\newacronym{NMSE}{NMSE}{Normalized Mean Square Error}
\newacronym{NORM}{NORM}{Normalized Projection-based Grouping}
\newacronym{NP}{NP}{Non-Polynomial Time}
\newacronym{NRT}{NRT}{Non-Real Time}
\newacronym{NSPS}{NSPS}{National Security and Public Safety Services}
\newacronym{O2I}{O2I}{Outdoor to Indoor}
\newacronym{OFDMA}{OFDMA}{Orthogonal Frequency Division Multiple Access}
\newacronym{OFDM}{OFDM}{Orthogonal Frequency Division Multiplexing}
\newacronym{OFPC}{OFPC}{Open Loop with Fractional Path Loss Compensation}
\newacronym{OL}{OL}{Open Loop}
\newacronym{OLPC}{OLPC}{Open-Loop Power Control}
\newacronym{OL-PC}{OL-PC}{Open-Loop Power Control}
\newacronym{OPEX}{OPEX}{Operational Expenditure}
\newacronym{ORB}{ORB}{Orthogonal Random Beamforming}
\newacronym{JO-PF}{JO-PF}{Joint Opportunistic Proportional Fair}
\newacronym{OSI}{OSI}{Open Systems Interconnection}
\newacronym{PAIR}{PAIR}{D2D Pair Gain-based Grouping}
\newacronym{PAPR}{PAPR}{Peak-to-Average Power Ratio}
\newacronym{P2P}{P2P}{Peer-to-Peer}
\newacronym{PBS}{PBS}{Pico Base Station}
\newacronym{PC}{PC}{Power Control}
\newacronym{PCI}{PCI}{Physical Cell ID}
\newacronym{PDCP}{PDCP}{Packet Data Convergence Protocol}
\newacronym{PDF}{PDF}{Probability Density Function}
\newacronym{PDN}{PDN}{Packet Data Network}
\newacronym{PER}{PER}{Packet Error Rate}
\newacronym{PLR}{PLR}{Packet Loss Ratio}
\newacronym{PF}{PF}{Proportional Fair}
\newacronym{P-GW}{P-GW}{Packet Data Network Gateway}
\newacronym{PHY}{PHY}{Physical}
\newacronym{PL}{PL}{Pathloss}
\newacronym{PRABE}{PRABE}{Power and Resource Allocation Based on Quality of Experience}
\newacronym{PRB}{PRB}{Physical Resource Block}
\newacronym{PROJ}{PROJ}{Projection-based Grouping}
\newacronym{ProSe}{ProSe}{Proximity Services}
\newacronym{PS}{PS}{Packet Scheduling}
\newacronym{PSO}{PSO}{Particle Swarm Optimization}
\newacronym{PTAS}{PTAS}{Polynomial-Time Approximation Scheme}
\newacronym{PZF}{PZF}{Projected Zero-Forcing}
\newacronym{QAM}{QAM}{Quadrature Amplitude Modulation}
\newacronym{QoS}{QoS}{Quality of Service}
\newacronym{QCI}{QCI}{\ac{QoS} Class Identifier}
\newacronym{QHMLWDF}{QHMLWDF}{Queue-HOL-MLWDF}
\newacronym{QoE}{QoE}{Quality of Experience}
\newacronym{QPSK}{QPSK}{Quadrature Phase Shift Keying}
\newacronym{QSM}{QSM}{Queue-based Satisfaction Maximization}
\newacronym{RAISES}{RAISES}{Reallocation-based Assignment for Improved Spectral Efficiency and Satisfaction}
\newacronym{RAN}{RAN}{Radio Access Network}
\newacronym{RA}{RA}{Resource Allocation}
\newacronym{RAT}{RAT}{Radio Access Technology}
\newacronym{RATE}{RATE}{Rate-based}
\newacronym{RB}{RB}{Resource Block}
\newacronym{RBG}{RBG}{Resource Block Group}
\newacronym{REF}{REF}{Reference Grouping}
\newacronym{RET}{RET}{Remote Electrical Tilt}
\newacronym{RF}{RF}{Radio Frequency}
\newacronym{RLC}{RLC}{Radio Link Control}
\newacronym{RM}{RM}{Rate Maximization}
\newacronym{RMEC}{RMEC}{Rate Maximization under Experience Constraints}
\newacronym{RNC}{RNC}{Radio Network Controller}
\newacronym{RND}{RND}{Random Grouping}
\newacronym{RRA}{RRA}{Radio Resource Allocation}
\newacronym{RRM}{RRM}{Radio Resource Management}
\newacronym{RSCP}{RSCP}{Received Signal Code Power}
\newacronym{RSRP}{RSRP}{Reference Signal Receive Power}
\newacronym{RSRQ}{RSRQ}{Reference Signal Receive Quality}
\newacronym{RR}{RR}{Round Robin}
\newacronym{RRC}{RRC}{Radio Resource Control}
\newacronym{RSSI}{RSSI}{Received Signal Strength Indicator}
\newacronym{RT}{RT}{Real Time}
\newacronym{RTS}{RTS}{Request to Send}
\newacronym{RU}{RU}{Resource Unit}
\newacronym{RUNE}{RUNE}{RUdimentary Network Emulator}
\newacronym{RV}{RV}{Random Variable}
\newacronym{RZF}{RZF}{Regularized Zero-Forcing}
\newacronym{SAC}{SAC}{Session Admission Control}
\newacronym{SAE}{SAE}{System Architecture Evolution}
\newacronym{SC}{SC}{Small Cell}
\newacronym{SCM}{SCM}{Spatial Channel Model}
\newacronym{SC-FDMA}{SC-FDMA}{Single Carrier - Frequency Division Multiple Access}
\newacronym{SD}{SD}{Soft Dropping}
\newacronym{S-D}{S-D}{Source-Destination}
\newacronym{SDPC}{SDPC}{Soft Dropping Power Control}
\newacronym{SDMA}{SDMA}{Space-Division Multiple Access}
\newacronym{SER}{SER}{Symbol Error Rate}
\newacronym{SES}{SES}{Simple Exponential Smoothing}
\newacronym{S-GW}{S-GW}{Serving Gateway}
\newacronym{SINR}{SINR}{Signal to Interference-plus-Noise Ratio}
\newacronym{SI}{SI}{Satisfaction Indicator}
\newacronym{SIP}{SIP}{Session Initiation Protocol}
\newacronym{SISO}{SISO}{Single Input Single Output}
\newacronym{SIMO}{SIMO}{Single Input Multiple Output}
\newacronym{SIR}{SIR}{Signal to Interference Ratio}
\newacronym{SLNR}{SLNR}{Signal-to-Leakage-plus-Noise Ratio}
\newacronym{SM}{SM}{Subcarrier Matching}
\newacronym{SMA}{SMA}{Simple Moving Average}
\newacronym{SNR}{SNR}{Signal to Noise Ratio}
\newacronym{SON}{SON}{Self Organizing Networks}
\newacronym{SORA}{SORA}{Satisfaction Oriented Resource Allocation}
\newacronym{SORA-NRT}{SORA-NRT}{Satisfaction-Oriented Resource Allocation for Non-Real Time Services}
\newacronym{SORA-RT}{SORA-RT}{Satisfaction-Oriented Resource Allocation for Real Time Services}
\newacronym{SPF}{SPF}{Single-Carrier Proportional Fair}
\newacronym{SRA}{SRA}{Sequential Removal Algorithm}
\newacronym{SRS}{SRS}{Sounding Reference Signal}
\newacronym{SU-MIMO}{SU-MIMO}{Single-User Multiple Input Multiple Output}
\newacronym{SU}{SU}{Single-User}
\newacronym{SVD}{SVD}{Singular Value Decomposition}
\newacronym{TCP}{TCP}{Transmission Control Protocol}
\newacronym{TDD}{TDD}{Time Division Duplex}
\newacronym{TDL}{TDL}{Tapped Delay Line}
\newacronym{TDMA}{TDMA}{Time Division Multiple Access}
\newacronym{TETRA}{TETRA}{Terrestrial Trunked Radio}
\newacronym{TP}{TP}{Transmit Power}
\newacronym{TPC}{TPC}{Transmit Power Control}
\newacronym{TTI}{TTI}{Transmission Time Interval}
\newacronym{TTR}{TTR}{Time-To-Rendezvous}
\newacronym{TSM}{TSM}{Throughput-based Satisfaction Maximization}
\newacronym{TU}{TU}{Typical Urban}
\newacronym{TV}{TV}{Television}
\newacronym{TVWS}{TVWS}{\acs{TV} White Space}
\newacronym{UE}{UE}{User Equipment}
\newacronym{UEPS}{UEPS}{Urgency and Efficiency-based Packet Scheduling}
\newacronym{UFC}{UFC}{Federal University of Cear\'{a}}
\newacronym{UL}{UL}{Uplink}
\newacronym{UMTS}{UMTS}{Universal Mobile Telecommunications System}
\newacronym{URI}{URI}{Uniform Resource Identifier}
\newacronym{URM}{URM}{Unconstrained Rate Maximization}
\newacronym{VET}{VET}{Variable Electrical Tilt}
\newacronym{VR}{VR}{Virtual Resource}
\newacronym{VoIP}{VoIP}{Voice over IP}
\newacronym{VTMLWDF}{VTMLWDF} {Virtual Token MLWDF}
\newacronym{WCDMA}{WCDMA}{Wideband Code Division Multiple Access}
\newacronym{WF}{WF}{Water-filling}
\newacronym{Wi-Fi}{Wi-Fi}{Wireless Fidelity}
\newacronym{WiMAX}{WiMAX}{Worldwide Interoperability for Microwave Access}
\newacronym{WINNER}{WINNER}{Wireless World Initiative New Radio}
\newacronym{WLAN}{WLAN}{Wireless Local Area Network}
\newacronym{WMPF}{WMPF}{Weighted Multicarrier Proportional Fair}
\newacronym{WP}{WP}{Work Package}
\newacronym{WPF}{WPF}{Weighted Proportional Fair}
\newacronym{WSN}{WSN}{Wireless Sensor Network}
\newacronym{WWW}{WWW}{World Wide Web}
\newacronym{XIXO}{XIXO}{(Single or Multiple) Input (Single or Multiple) Output}
\newacronym{ZF}{ZF}{Zero-Forcing}
\newacronym{ZMCSCG}{ZMCSCG}{Zero Mean Circularly Symmetric Complex Gaussian}

% ---
% Pacotes de citações
%\usepackage[num,abnt-etal-list=3,abnt-doi=link]{abntex2cite}
%\citebrackets[]
\usepackage[brazilian,hyperpageref]{backref}	 % Paginas com as citações na bibl
\usepackage[alf,abnt-and-type=e,abnt-full-initials=no,abnt-last-names=abnt,abnt-etal-list=2,abnt-etal-text = emph,abnt-emphasize=bf]{abntex2cite}

% ---
% Configurações de aparência do PDF final

\renewcommand{\cftsubsectionfont}{\itshape}
\renewcommand{\cftsubsubsectionfont}{\itshape}
\renewcommand{\cftsubsubsectionfont}{\cftsubsectionfont}


% alterando o aspecto da cor azul
\definecolor{blue}{RGB}{41,5,195}

% informações do PDF
\makeatletter
\hypersetup{
     	%pagebackref=true,
		pdftitle={\@title}, 
		pdfauthor={\@author},
    	pdfsubject={\imprimirpreambulo},
	    pdfcreator={LaTeX with abnTeX2},
		pdfkeywords={abnt}{latex}{abntex}{abntex2}{trabalho acadêmico}, 
		colorlinks=true,       		% false: boxed links; true: colored links
    	linkcolor=black,          	% color of internal links
    	citecolor=black,        		% color of links to bibliography
    	filecolor=black,      		% color of file links
		urlcolor=black,
		bookmarksdepth=4
}
\makeatother
% --- 

\usepackage{lipsum}				% para geração de dummy text

% Definições matemáticas
\newcommand{\EqRef}[2][]{eq.#1~\eqref{#2}}
\newcommand{\AlgRef}[2][]{Alg.#1~\ref{#2}}
\newcommand{\mean}[1]{\mathbb{E}\left\{ #1 \right\}}
\newcommand{\Abs}[1]{\left\vert #1 \right\vert}
\newcommand\figref{Figura~\ref}
\newcommand\capref{Capítulo~\ref}
\newcommand\tabref{Tabela~\ref}
\newcommand\secref{Seção~\ref}
\def\nTTI{[n]}
\newcommand{\stK}{\Set{K}}
\usepackage{mathtools}
\usepackage{amssymb,amstext,amsfonts}

\addto\captionsenglish{%
	\renewcommand{\listfigurename}{LISTA DE FIGURAS}%
	\renewcommand{\listtablename}{LISTA DE TABELAS}%
	\renewcommand{\bibname}{BIBLIOGRAFIA}%
	\renewcommand{\figurename}{Figura}%
	\renewcommand{\tablename}{Tabela}%
	\renewcommand{\contentsname}{SUMÁRIO}%			
}


% Pacote para tabela
\usepackage{threeparttable}

% Cross reference
\usepackage{xr}
% --- 
% Espaçamentos entre linhas e parágrafos 
% --- 

\usepackage{tocloft}

%%%%%%%%%%%%%%%%%%%%%%%%%%%%%%%%%%%%%%%%%%%%%%%%%%%%%%%%%%%%%%%%
%%% Para incluir figuras usando tikz/pdfplots
\RequirePackage{tikz}% Create graphics in Latex
% Definitions and libraries for the Tikz package

% xxxxx Libraries xxxxxxxxxxxxxxxxxxxxxxxxxxxxxxxxxxxxxxxxxxxxxxxxxxxxxxxxx
\usetikzlibrary{positioning}
\usetikzlibrary{calc,arrows}
\usetikzlibrary{matrix}
\usetikzlibrary{fit}
\usetikzlibrary{decorations.pathreplacing}
\usetikzlibrary{shapes}
\usetikzlibrary{shadows}
\usetikzlibrary{patterns}
\usetikzlibrary{chains}
%\usetikzlibrary{external}
%\tikzexternalize[prefix=Figures/PDFtikz/]

% xxxxx Layers xxxxxxxxxxxxxxxxxxxxxxxxxxxxxxxxxxxxxxxxxxxxxxxxxxxxxxxxxxxx
\pgfdeclarelayer{below background}
\pgfdeclarelayer{background layer}
\pgfdeclarelayer{foreground layer}
\pgfsetlayers{below background,background layer,main,foreground layer}
% xxxxxxxxxxxxxxxxxxxxxxxxxxxxxxxxxxxxxxxxxxxxxxxxxxxxxxxxxxxxxxxxxxxxxxxxx
 % Load libraries and styles for the Tikz Package

\RequirePackage{pgfplots}
%%%%%%%%%%%%%%%%%%%%%%%%%%%%%%%%%%%%%%%%%%%%%%%%%%%%%%%%%%%%%%%%

% O tamanho do parágrafo é dado por:
\setlength{\parindent}{1.3cm}

% Controle do espaçamento entre um parágrafo e outro:
\setlength{\parskip}{0.2cm}  % tente também \onelineskip

% ---
% compila o indice
% ---
\makeindex
% ---