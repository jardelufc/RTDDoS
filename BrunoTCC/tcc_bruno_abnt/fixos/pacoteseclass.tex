\documentclass[
	% -- opções do pacote babel --
	english,		% idioma adicional para hifenização
	portuguese,			% o último idioma é o principal do documento
	% -- opções da classe memoir --
	12pt,				% tamanho da fonte
	openright,			% capítulos começam em pág ímpar (insere página vazia caso preciso)
	oneside,			% para impressão em verso e anverso. Oposto a oneside
	a4paper,			% tamanho do papel. 
	% -- opções da classe abntex2 --
	%chapter=TITLE,		% títulos de capítulos convertidos em letras maiúsculas
	%section=TITLE,		% títulos de seções convertidos em letras maiúsculas
	%subsection=TITLE,	% títulos de subseções convertidos em letras maiúsculas
	%subsubsection=TITLE,% títulos de subsubseções convertidos em letras maiúsculas
	% -- opções de sumário -- %
	sumario=abnt-6027-2012 % sumário conforme as recomendações da ABNT NBR 6027:2012 (padrão)
%	sumario=tradicional, % usa o estilo tradicional de sumários do memoir
	%	.	
	]{abntex2}

% Modificacoes pessoais de acordo com a norma da UFC (?)
\renewcommand{\ABNTEXchapterfontsize}{\bfseries\normalsize}
\renewcommand{\ABNTEXchapterfont}{\normalfont}
\setboolean{ABNTEXupperchapter}{true}
\renewcommand{\ABNTEXsectionfontsize}{\bfseries\normalsize}
\renewcommand{\ABNTEXsubsectionfontsize}{\bfseries\itshape\normalsize}
\renewcommand{\ABNTEXsubsubsectionfontsize}{\itshape\normalsize}
\renewcommand{\ABNTEXsubsubsubsectionfontsize}{\normalsize}

\renewcommand{\cftfigurename}{Figura\space}
\renewcommand{\cfttablename}{Tabela\space}

\setlength{\cftbeforechapterskip}{0em}

% --- 
% CONFIGURAÇÕES DE PACOTES
% --- 

%%%%%%%%%%%%%%%%%%%%%%%%%%%%%%%%%%%%%%%%%%%%%%%%%%%%%%%%%%%%%%%%
%%% I had to include this package here because there was an
%%% incompatibility with another package
\usepackage[svgnames]{xcolor}
%%%%%%%%%%%%%%%%%%%%%%%%%%%%%%%%%%%%%%%%%%%%%%%%%%%%%%%%%%%%%%%%

\usepackage{pdfpages}

\usepackage{pbox}
% ---
% Pacotes básicos 
% ---

\usepackage{babel}
\usepackage{lmodern}			% Usa a fonte Latin Modern			
\usepackage[T1]{fontenc}		% Selecao de codigos de fonte.
\usepackage[utf8]{inputenc}		% Codificacao do documento (conversão automática dos acentos)
\usepackage{lastpage}			% Usado pela Ficha catalográfica
\usepackage{indentfirst}		% Indenta o primeiro parágrafo de cada seção.
\usepackage{color}				% Controle das cores
\newcommand{\ownership}{\flushleft\footnotesize Fonte: Elaborada pelo autor.}
\newcommand{\source}{\flushleft\footnotesize Fonte: }
\usepackage{microtype} 			% para melhorias de justificação
\usepackage{ufc-abntex2}

%%%%%%%%%%%%%%%%%%%%%%%%%%%%%%%%%%%%%%%%%%%%%%%%%%%%%%%%%%%%%%%%
%%% Para incluir gráficos -- Formato .eps
\graphicspath{{.}{./Figures/}}
\DeclareGraphicsExtensions{.pdf,.eps}
%%%%%%%%%%%%%%%%%%%%%%%%%%%%%%%%%%%%%%%%%%%%%%%%%%%%%%%%%%%%%%%%

%%%%%%%%%%%%%%%%%%%%%%%%%%%%%%%%%%%%%%%%%%%%%%%%%%%%%%%%%%%%
%%% Incluir pacote para gerenciar espaço entre valores e unidades
\usepackage{siunitx}
\sisetup{%
	binary-units,
	per-mode = symbol,
	detect-all,
	list-units = single,
	list-final-separator = { and },
	%range-phrase={--},
	range-units = single,
	round-mode = places,
	round-precision = 2,
	product-units = single,
}
%%%%%%%%%%%%%%%%%%%%%%%%%%%%%%%%%%%%%%%%%%%%%%%%%%%%%%%%%%%%

% Maths
\usepackage{amsthm}
\usepackage{amsmath,dsfont,xfrac,bm}            
% Organizar numeração das figuras por capítulo 
\numberwithin{figure}{chapter}
% Organizar numeração das tabelas por capítulo
\numberwithin{table}{chapter}

\usepackage{subcaption}

\usepackage{multirow}

%%%%%%%%%%%%%%%%%%%%%%%%%%%%%%%%%%%%%%%%%%%%%%%%%%%%%%%%%%%%
% Package algorithm -- pacote para manipulação de pseudo-código.
\usepackage{algorithm}
\usepackage{algorithmicx}
\usepackage{algpseudocode}

%%%%%%%%%%%%%%%%%%%%%%%%%%%%%%%%%%%%%%%%%%%%%%%%%%%%%%%%%%%%
%%% Package enumerare -- Extensão do ambiente "enumerate".
\usepackage{enumerate}
		
% ---
% Pacotes adicionais, usados apenas no âmbito do Modelo Canônico do abnteX2
% ---

% Acronyms
%\usepackage[nomain,acronym,toc,shortcuts]{glossaries}
\usepackage[nogroupskip,acronym,shortcuts]{glossaries}
\newglossary{symbols}{sym}{sbl}{\symbname} % glossary for symbols
\makenoidxglossaries

% Style witlonder length for description https://tex.stackexchange.com/questions/21666/glossary-term-list-width
\newglossarystyle{clong}{%
	\renewenvironment{theglossary}%
	{\begin{longtable}{lp{1.5\glsdescwidth}}}%
		{\end{longtable}}%
	\renewcommand*{\glossaryheader}{}%
	\renewcommand*{\glsgroupheading}[1]{}%
	\renewcommand*{\glossaryentryfield}[5]{%
		\glstarget{##1}{##2} & ##3\glspostdescription\space ##5\\}%
	\renewcommand*{\glossarysubentryfield}[6]{%
		& \glstarget{##2}{\strut}##4\glspostdescription\space ##6\\}%
	%\renewcommand*{\glsgroupskip}{ & \\}%
}
%\chapter*{Acronyms}
%\renewcommand{\titulonome}{Acronyms}%
%\renewcommand{\prepbynome}{UFC.33 Team}%

%\begin{singlespace}
\begin{acronym}[LTE-Advanced]%\addtolength{\itemsep}{-0.5\baselineskip}
	\acro{2G}{Second Generation}
	\acro{3G}{3ª Geração}
	%  \acro{3GPP}{3$^\text{rd}$~Generation Partnership Project}
	\acro{3GPP}{3rd Generation Partnership Project}
	\acro{3GPP2}{3rd Generation Partnership Project 2}
	\acro{1G}{1ª Geração}
	\acro{4G}{4ª Geração}
	\acro{5G}{5ª Geração}
	\acro{AA}{Antenna Array}
	\acro{AC}{Admission Control}
	\acro{AD}{Attack-Decay}
	\acro{ABS}{Almost Blank Subframe}
	\acro{ADSL}{Asymmetric Digital Subscriber Line}
	\acro{AHW}{Alternate Hop-and-Wait}
	\acro{AMC}{Adaptive Modulation and Coding}
	\acro{AP}{Access Point}
	\acro{APA}{Adaptive Power Allocation}
	\acro{ARMA}{Autoregressive Moving Average}
	\acro{ASC}{Adaptive Satisfaction Control}
	\acro{ATES}{Adaptive Throughput-based Efficiency-Satisfaction Trade-Off}
	\acro{AWGN}{Additive White Gaussian Noise}
	\acro{BB}{Branch and Bound}
	\acro{BC}{Branch and Cut}
	\acro{BD}{Block Diagonalization}
	\acro{BER}{Bit Error Rate}
	\acro{BF}{Best Fit}
	\acro{BL}{Bit Loading}
	\acro{BLER}{BLock Error Rate}
	\acro{BPC}{Binary Power Control}
	\acro{BPSK}{Binary Phase-Shift Keying}
	\acro{BRA}{Balanced Random Allocation}
	\acro{BS}{Base Station}
	\acro{BSP}{\ac{BS} Placement}
	\acro{CAP}{Combinatorial Allocation Problem}
	\acro{CAPEX}{Capital Expenditure}
	\acro{CBF}{Coordinated Beamforming}
	\acro{CBR}{Constant Bit Rate}
	\acro{CBS}{Class Based Scheduling}
	\acro{CC}{Congestion Control}
	\acro{CDF}{Cumulative Distribution Function}
	\acro{CDMA}{\textit{Code-Division Multiple Access}}
	\acro{C-RAN}{Cloud - Based Radio Access Network}
	\acro{CL}{Closed Loop}
	\acro{CLPC}{Closed Loop Power Control}    
	\acro{CNR}{Channel-to-Noise Ratio}
	\acro{CPA}{Cellular Protection Algorithm}
	\acro{CPICH}{Common Pilot Channel}
	\acro{CoMP}{Coordinated Multi-Point}
	\acro{CQI}{Channel Quality Indicator}
	\acro{CRE}{Cell Range Expansion}
	\acro{CRM}{Constrained Rate Maximization}
	\acro{CRN}{Cognitive Radio Network}
	\acro{CS}{Coordinated Scheduling}
	\acro{CSI}{Channel State Information}
	\acro{CTS}{Clear to Send}
	\acro{CUE}{Cellular User Equipment}
	\acro{D2D}{Device-to-Device}
	\acro{DCA}{Dynamic Channel Allocation}
	\acro{DE}{Differential Evolution}
	\acro{DF}{Decode and Forward}
	\acro{DFT}{Discrete Fourier Transform}
	%  \acro{DIST}{Distance-based Grouping}
	\acro{DIST}{Distance}
	\acro{DL}{Downlink}
	\acro{DMA}{Double Moving Average}
	\acro{DMRS}{Demodulation Reference Signal}
	\acro{D2DM}{D2D Mode}
	\acro{DMS}{D2D Mode Selection}
	\acro{DPC}{Dirty Paper Coding}
	\acro{DRA}{Dynamic Resource Assignment}
	\acro{DSA}{Dynamic Spectrum Access}
	\acro{DSM}{\textit{Delay-based Satisfaction Maximization}}
	\acro{ECC}{Electronic Communications Committee}
	\acro{EE}{Energy Efficiency}
	\acro{EFLC}{Error Feedback Based Load Control}
	\acro{EI}{Efficiency Indicator}
	\acro{e-ICIC}{Enhanced Inter-Cell Interference Coordination}
	\acro{eNB}{Evolved Node B}
	\acro{EXP}{\textit{Exponential}}
	\acro{EPA}{Equal Power Allocation}
	\acro{EPC}{Evolved Packet Core}
	\acro{EPS}{Evolved Packet System}
	\acro{E-UTRAN}{Evolved Universal Terrestrial Radio Access Network}
	\acro{ES}{Exhaustive Search}
	\acro{FCP}{Fundamental Counting Principle}
	\acro{FD}{Full-Duplex Communications}
	\acro{FDD}{Frequency Division Duplex}
	\acro{FDM}{Frequency Division Multiplexing}
	\acro{FDMA}{Frequency Division Multiple Access}
	\acro{FER}{Frame Erasure Rate}
	\acro{FF}{Fast Fading}
	\acro{FSB}{Fixed Switched Beamforming}
	\acro{FST}{Fixed SNR Target}
	\acro{FTP}{File Transfer Protocol}
	\acro{GA}{Genetic Algorithm}
	\acro{GAP}{Generalized Assignment Problem}
	\acro{GAP-MQ}{Generalized Assignment Problem with Minimum Quantities}
	\acro{GATES}{Generalized Adaptive Throughput-based Efficiency-Satisfaction Trade-Off}
	\acro{GBR}{Guaranteed Bit Rate}
	\acro{GLR}{Gain to Leakage Ratio}
	\acro{GOS}{Generated Orthogonal Sequence}
	\acro{GPL}{GNU General Public License}
	\acro{GPS}{Global Positioning System}
	\acro{GRP}{Grouping}
	\acro{GTEL}{Wireless Telecommunications Research Group}
	\acro{HARQ}{Hybrid Automatic Repeat Request}
	\acro{HCPP}{Hardcore Point Process}
	\acro{HD}{High Definition}
	\acro{HetNet}{Heterogeneous Network}
	\acro{HH}{Hughes-Hartogs}
	\acro{HMS}{Harmonic Mode Selection}
	\acro{HOL}{Head Of Line}
	\acro{HSDPA}{High-Speed Downlink Packet Access}
	\acro{HSPA}{High Speed Packet Access}
	\acro{HTTP}{HyperText Transfer Protocol}
	\acro{ICMP}{Internet Control Message Protocol} 
	\acro{ICI}{Intercell Interference}
	\acro{ICIC}{Inter-Cell Interference Coordination}
	\acro{ID}{Identification}
	\acro{IETF}{Internet Engineering Task Force}
	%\acro{ILP}{Integer Linear Program}
	\acro{UID}{Unique Identification}
	\acro{IID}{Independent and Identically Distributed}
	\acro{IIR}{Infinite Impulse Response}
	\acro{ILP}{Integer Linear Problem}
	\acro{IMT}{International Mobile Telecommunications}
	\acro{INV}{Inverted Norm-based Grouping} 
	\acro{IoT}{Internet of Things}
	\acro{IP}{Internet Protocol}
	\acro{IPv6}{Internet Protocol Version 6}
	\acro{ISD}{Inter-Site Distance}
	\acro{ISI}{Inter Symbol Interference}
	\acro{ISM}{Industrial, Scientific and Medical}
	\acro{ITU}{International Telecommunication Union}
	\acro{JOAS}{Joint Opportunistic Assignment and Scheduling}
	\acro{JOS}{Joint Opportunistic Scheduling}
	\acro{JP}{Joint Processing}
	\acro{JRAPAP}{Joint RB Assignment and Power Allocation Problem}
	\acro{JS}{Jump-Stay}
	\acro{JSM}{Joint Satisfaction Maximization}
	\acro{KKT}{Karush-Kuhn-Tucker}
	\acro{LAC}{Link Admission Control}
	\acro{LA}{Link Adaptation}
	\acro{LC}{Load Control}
	\acro{LOS}{Line of Sight}
	\acro{LP}{Linear Programming}
	\acro{LTE}{Long Term Evolution}
	\acro{LTE-A}{\ac{LTE}-Advanced}
	\acro{LTE-Advanced}{Long Term Evolution Advanced}
	\acro{M2M}{Machine-to-Machine}
	\acro{MAC}{Medium Access Control}
	\acro{MANET}{Mobile Ad hoc Network}
	\acro{MEDS}{Method of Exact Doppler Spread}
	\acro{MC}{Modular Clock}
	\acro{MCS}{Modulation and Coding Scheme}
	\acro{MDB}{Measured Delay Based}
	\acro{MDI}{Minimum D2D Interference}
	\acro{MDSM}{Modified Delay-based Satisfaction Maximization}
	\acro{MDU} {\textit{Max-Delay-Utility}}
	\acro{MF}{Matched Filter}
	\acro{MG}{Maximum Gain}
	\acro{MH}{Multi-Hop}
	\acro{MILP}{Mixed Integer Linear Programming}
	\acro{MIMO}{Multiple Input Multiple Output}
	\acro{MINLP}{Mixed Integer Nonlinear Programming}
	\acro{MIP}{Mixed Integer Programming}
	\acro{MISO}{Multiple Input Single Output}
	\acro{MLWDF}{\textit{Modified Largest Weighted Delay First}}
	\acro{MME}{Mobility Management Entity}
	\acro{MMF}{\textit{Max}-\textit{Min Fairness}}
	\acro{MMSE}{Minimum Mean Square Error}
	\acro{mmW}{Millimeter Wave}
	\acro{MOS}{Mean Opinion Score}
	\acro{MPF}{Multicarrier Proportional Fair}
	\acro{MPRP}{Maximization of the Product of the Residual Powers}
	\acro{MRA}{Maximum Rate Allocation}
	\acro{MR}{Maximum Rate}
	\acro{MRC}{Maximum Ratio Combining}
	\acro{MRT}{Maximum Ratio Transmission}
	\acro{MRUS}{Maximum Rate with User Satisfaction}
	\acro{MS}{Mode Selection}
	\acro{MSE}{Mean Squared Error}
	\acro{MSI}{Multi-Stream Interference}
	\acro{MTC}{Machine-Type Communication}
	\acro{MTSI}{Multimedia Telephony Services over IMS}
	\acro{MTSM}{Modified Throughput-based Satisfaction Maximization}
	\acro{MU-MIMO}{Multi-User Multiple Input Multiple Output}
	\acro{MU}{Multi-User}
	\acro{Multi-CUT}{Multi-Cell and Multi-User and Multi-Tier}
	\acro{NAS}{Non-Access Stratum}
	\acro{NB}{Node B}
	\acro{NCL}{Neighbor Cell List}
	\acro{NLP}{Nonlinear Programming}
	\acro{NLOS}{Non-Line of Sight}
	\acro{NMSE}{Normalized Mean Square Error}
	\acro{NORM}{Normalized Projection-based Grouping}
	\acro{NP}{Non-Polynomial Time}
	\acro{NRT}{Non-Real Time}
	\acro{NSPS}{National Security and Public Safety Services}
	%\acro{O2I}{Outdoor to Indoor}
	\acro{OFDMA}{Orthogonal Frequency Division Multiple Access}
	\acro{OFDM}{Orthogonal Frequency Division Multiplexing}
	\acro{OFPC}{Open Loop with Fractional Path Loss Compensation}
	\acro{O2I}{Outdoor-to-Indoor}
	\acro{OL}{Open Loop}
	\acro{OLPC}{Open-Loop Power Control}
	\acro{OL-PC}{Open-Loop Power Control}
	\acro{OPEX}{Operational Expenditure}
	\acro{ORB}{Orthogonal Random Beamforming}
	\acro{JO-PF}{Joint Opportunistic Proportional Fair}
	\acro{OSI}{Open Systems Interconnection}
	\acro{PAIR}{D2D Pair Gain-based Grouping}
	\acro{PAPR}{Peak-to-Average Power Ratio}
	\acro{P2P}{Peer-to-Peer}
	\acro{PBS}{Pico Base Station}        
	\acro{PC}{Power Control}
	\acro{PCI}{Physical Cell ID}
	\acro{PDF}{Probability Density Function}
	\acro{PER}{Packet Error Rate}
	\acro{PF}{\textit{Proportional Fair}}
	\acro{P-GW}{Packet Data Network Gateway}
	\acro{PHY}{Physical}
	\acro{PL}{Pathloss}
	\acro{PLR}{Packet Loss Ratio}
	\acro{PRABE}{Power and Resource Allocation Based on Quality of Experience}
	\acro{PRB}{Physical Resource Block}
	\acro{PROJ}{Projection-based Grouping}
	\acro{ProSe}{Proximity Services}
	\acro{PS}{Packet Scheduling}
	\acro{PSO}{Particle Swarm Optimization}
	\acro{PTAS}{Polynomial-Time Approximation Scheme}
	\acro{PZF}{Projected Zero-Forcing}
	\acro{QAM}{Quadrature Amplitude Modulation}
	\acro{QHMLWDF}{\textit{Queue}-HOL-MLWDF}
	\acro{QoE}{Quality of Experience}
	\acro{QoS}{Quality of Service}
	\acro{QPSK}{Quadri-Phase Shift Keying}
	\acro{RAISES}{Reallocation-based Assignment for Improved Spectral Efficiency and Satisfaction}
	\acro{RAN}{Radio Access Network}
	\acro{RA}{Resource Allocation}
	\acro{RAT}{Radio Access Technology}
	\acro{RATE}{Rate-based}
	\acro{RB}{Resource Block}
	\acro{RBG}{Resource Block Group}
	\acro{REF}{Reference Grouping}
	\acro{RET}{Remote Electrical Tilt}
	\acro{RF}{Radio Frequency}
	\acro{RLC}{Radio Link Control}
	\acro{RM}{\textit{Rate Maximization}}
	\acro{RMEC}{Rate Maximization under Experience Constraints}
	\acro{RNC}{Radio Network Controller}
	\acro{RND}{Random Grouping}
	\acro{RRA}{Radio Resource Allocation}
	\acro{RRM}{Radio Resource Management}
	\acro{RSCP}{Received Signal Code Power}
	\acro{RSRP}{Reference Signal Receive Power}
	\acro{RSRQ}{Reference Signal Receive Quality}
	\acro{RR}{Round Robin}
	\acro{RRC}{Radio Resource Control}
	\acro{RSSI}{Received Signal Strength Indicator}
	\acro{RT}{Real Time}
	\acro{RTS}{Request to Send}
	\acro{RU}{Resource Unit}
	\acro{RUNE}{RUdimentary Network Emulator}
	\acro{RV}{Random Variable}
	\acro{RZF}{Regularized Zero-Forcing}
	\acro{SAC}{Session Admission Control}
	\acro{SC}{Small Cell}
	\acro{SCM}{Spatial Channel Model}
	\acro{SC-FDMA}{Single Carrier - Frequency Division Multiple Access}
	\acro{SD}{Soft Dropping}
	\acro{S-D}{Source-Destination}
	\acro{SDPC}{Soft Dropping Power Control}
	\acro{SDMA}{Space-Division Multiple Access}
	\acro{SER}{Symbol Error Rate}
	\acro{SES}{Simple Exponential Smoothing}
	\acro{S-GW}{Serving Gateway}
	\acro{SINR}{Signal to Interference-plus-Noise Ratio}
	\acro{SI}{Satisfaction Indicator}
	\acro{SIP}{Session Initiation Protocol}
	\acro{SISO}{\textit{Single Input Single Output}}
	\acro{SIMO}{Single Input Multiple Output}
	\acro{SIR}{Signal to Interference Ratio}
	\acro{SLNR}{Signal-to-Leakage-plus-Noise Ratio}
	\acro{SM}{Subcarrier Matching}
	\acro{SMA}{Simple Moving Average}
	\acro{SNR}{Signal to Noise Ratio}
	\acro{SON}{Self Organizing Networks}
	\acro{SORA}{Satisfaction Oriented Resource Allocation}
	\acro{SORA-NRT}{\textit{Satisfaction-Oriented Resource Allocation for Non-Real Time Services}}
	\acro{SORA-RT}{\textit{Satisfaction-Oriented Resource Allocation for Real Time Services}}
	\acro{SPF}{Single-Carrier Proportional Fair}
	\acro{SRA}{Sequential Removal Algorithm}
	\acro{SRS}{Sounding Reference Signal}
	\acro{SU-MIMO}{Single-User Multiple Input Multiple Output}
	\acro{SU}{Single-User}
	\acro{SVD}{Singular Value Decomposition}
	\acro{TCP}{Transmission Control Protocol}
	\acro{TDD}{Time Division Duplex}
	\acro{TDMA}{Time Division Multiple Access}
	\acro{TETRA}{Terrestrial Trunked Radio}
	\acro{TP}{Transmit Power}
	\acro{TPC}{Transmit Power Control}
	\acro{TTI}{Transmission Time Interval}
	\acro{TTR}{Time-To-Rendezvous}
	\acro{TSM}{\textit{Throughput-based Satisfaction Maximization}}
	\acro{TU}{Typical Urban}
	\acro{TV}{Television}
	\acro{TVWS}{\acs{TV} White Space}
	\acro{UE}{User Equipment}
	\acro{UEPS}{\textit{Urgency and Efficiency-based Packet Scheduling}}
	\acro{UFC}{Federal University of Cear\'{a}}
	\acro{UL}{Uplink}
	\acro{UMTS}{Universal Mobile Telecommunications System}
	\acro{URI}{Uniform Resource Identifier}
	\acro{URM}{Unconstrained Rate Maximization}
	\acro{VBR}{Variable Bit Rate}
	\acro{VET}{Variable Electrical Tilt}
	\acro{VR}{Virtual Resource}
	\acro{VoIP}{Voice over IP}
	\acro{VTMLWDF} {\textit{Virtual Token} MLWDF}
	\acro{WCDMA}{Wideband Code Division Multiple Access}
	\acro{WF}{Water-filling}
	\acro{Wi-Fi}{Wireless Fidelity}
	\acro{WiMAX}{Worldwide Interoperability for Microwave Access}
	\acro{WINNER}{Wireless World Initiative New Radio}
	\acro{WLAN}{Wireless Local Area Network}
	\acro{WMPF}{Weighted Multicarrier Proportional Fair}
	\acro{WP}{Work Package}
	\acro{WPF}{Weighted Proportional Fair}
	\acro{WSN}{Wireless Sensor Network}
	\acro{WWW}{World Wide Web}
	\acro{XIXO}{(Single or Multiple) Input (Single or Multiple) Output}
	\acro{ZF}{Zero-Forcing}
	\acro{ZMCSCG}{\textit{Zero Mean Circularly Symmetric Complex Gaussian}}
\end{acronym}
%\end{singlespace}



% ---
% Pacotes de citações
%\usepackage[num,abnt-etal-list=3,abnt-doi=link]{abntex2cite}
%\citebrackets[]
\usepackage[brazilian,hyperpageref]{backref}	 % Paginas com as citações na bibl
\usepackage[alf,abnt-and-type=e,abnt-full-initials=no,abnt-last-names=abnt,abnt-etal-list=2,abnt-etal-text = emph,abnt-emphasize=bf]{abntex2cite}

% ---
% Configurações de aparência do PDF final

\renewcommand{\cftsubsectionfont}{\itshape}
\renewcommand{\cftsubsubsectionfont}{\itshape}
\renewcommand{\cftsubsubsectionfont}{\cftsubsectionfont}


% alterando o aspecto da cor azul
\definecolor{blue}{RGB}{41,5,195}

% informações do PDF
\makeatletter
\hypersetup{
     	%pagebackref=true,
		pdftitle={\@title}, 
		pdfauthor={\@author},
    	pdfsubject={\imprimirpreambulo},
	    pdfcreator={LaTeX with abnTeX2},
		pdfkeywords={abnt}{latex}{abntex}{abntex2}{trabalho acadêmico}, 
		colorlinks=true,       		% false: boxed links; true: colored links
    	linkcolor=black,          	% color of internal links
    	citecolor=black,        		% color of links to bibliography
    	filecolor=black,      		% color of file links
		urlcolor=black,
		bookmarksdepth=4
}
\makeatother
% --- 

\usepackage{lipsum}				% para geração de dummy text

% Definições matemáticas
\newcommand{\EqRef}[2][]{eq.#1~\eqref{#2}}
\newcommand{\AlgRef}[2][]{Alg.#1~\ref{#2}}
\newcommand{\mean}[1]{\mathbb{E}\left\{ #1 \right\}}
\newcommand{\Abs}[1]{\left\vert #1 \right\vert}
\newcommand\figref{Figura~\ref}
\newcommand\capref{Capítulo~\ref}
\newcommand\tabref{Tabela~\ref}
\newcommand\secref{Seção~\ref}
\def\nTTI{[n]}
\newcommand{\stK}{\Set{K}}
\usepackage{mathtools}
\usepackage{amssymb,amstext,amsfonts}

\addto\captionsenglish{%
	\renewcommand{\listfigurename}{LISTA DE FIGURAS}%
	\renewcommand{\listtablename}{LISTA DE TABELAS}%
	\renewcommand{\bibname}{BIBLIOGRAFIA}%
	\renewcommand{\figurename}{Figura}%
	\renewcommand{\tablename}{Tabela}%		
}


% Pacote para tabela
\usepackage{threeparttable}

% Cross reference
\usepackage{xr}
% --- 
% Espaçamentos entre linhas e parágrafos 
% --- 

\usepackage{tocloft}

%%%%%%%%%%%%%%%%%%%%%%%%%%%%%%%%%%%%%%%%%%%%%%%%%%%%%%%%%%%%%%%%
%%% Para incluir figuras usando tikz/pdfplots
\RequirePackage{tikz}% Create graphics in Latex
% Definitions and libraries for the Tikz package

% xxxxx Libraries xxxxxxxxxxxxxxxxxxxxxxxxxxxxxxxxxxxxxxxxxxxxxxxxxxxxxxxxx
\usetikzlibrary{positioning}
\usetikzlibrary{calc,arrows}
\usetikzlibrary{matrix}
\usetikzlibrary{fit}
\usetikzlibrary{decorations.pathreplacing}
\usetikzlibrary{shapes}
\usetikzlibrary{shadows}
\usetikzlibrary{patterns}
\usetikzlibrary{chains}
%\usetikzlibrary{external}
%\tikzexternalize[prefix=Figures/PDFtikz/]

% xxxxx Layers xxxxxxxxxxxxxxxxxxxxxxxxxxxxxxxxxxxxxxxxxxxxxxxxxxxxxxxxxxxx
\pgfdeclarelayer{below background}
\pgfdeclarelayer{background layer}
\pgfdeclarelayer{foreground layer}
\pgfsetlayers{below background,background layer,main,foreground layer}
% xxxxxxxxxxxxxxxxxxxxxxxxxxxxxxxxxxxxxxxxxxxxxxxxxxxxxxxxxxxxxxxxxxxxxxxxx
 % Load libraries and styles for the Tikz Package

\RequirePackage{pgfplots}
%%%%%%%%%%%%%%%%%%%%%%%%%%%%%%%%%%%%%%%%%%%%%%%%%%%%%%%%%%%%%%%%

% O tamanho do parágrafo é dado por:
\setlength{\parindent}{1.3cm}

% Controle do espaçamento entre um parágrafo e outro:
\setlength{\parskip}{0.2cm}  % tente também \onelineskip

% ---
% compila o indice
% ---
\makeindex
% ---