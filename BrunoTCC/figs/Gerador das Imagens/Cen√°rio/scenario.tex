% Document class like in main document
\documentclass[%
  %aspectratio=1610,
  english,
  %serif,
  %compress,
  %11pt,
  8pt,
  xcolor={table,svgnames},
  usepdftitle=false,
  hyperref={breaklinks=true}, % break long section names in ToC
  %t, % top alignment for all structures (e.g., frames, columns, etc)
  ]{beamer}

% The theme of GTEL
\usepackage{gtel-theme}

% Encoding
%\usepackage[utf8]{inputenc}
%\usepackage[T1]{fontenc}

% Fonts
%\usepackage[cmex10]{mathtools}
%\usepackage{newtxtext,newtxmath}
\usepackage{mathtools}
\usepackage{helvet}
\usepackage[helvet]{sfmath}
\renewcommand{\familydefault}{\sfdefault}
\usepackage{pifont}
%\usepackage{arev}

% More with fonts
\usepackage{relsize}
\usepackage{anyfontsize}

% Compile with xelatex
%\usepackage[no-math]{fontspec}
%\setmainfont{Arial}
%\usepackage{mathastext}
%%\Mathastext[Arial]

\usepackage{microtype}

% Picture size
%\newcommand{\defLength}[2]{\newlength{#1}\setlength{#1}{#2}}
%\defLength{\textWidth}{469.47049pt} % text width in main document
%\defLength{\textHeight}{674.33032pt} % text height in main document
%\defLength{\textWidth}{252pt}
%\defLength{\textHeight}{672pt}
%\usepackage[textwidth=\textWidth,height=\textHeight]{geometry}

% The drawing
\usepackage{tikz}
\usetikzlibrary{calc}
\usetikzlibrary{shapes.geometric}
\usetikzlibrary{fit}
\usetikzlibrary{positioning}
\usetikzlibrary{backgrounds}
\usetikzlibrary{arrows}
\usetikzlibrary{matrix}

\newcounter{dummy}
\def\getNoOfElements#1#2{%
  \setcounter{dummy}{0}% 
  \foreach\dummy in #1 {\stepcounter{dummy}}%
  \edef#2{\arabic{dummy}}
}

\def\hexasize{50mm}
\def\raio{4pt}
\def\arrowsize{12pt}
\def\lshift{2pt}
%\def\lstart{\lshift-\pgflinewidth}
\def\lstart{\lshift}

% http://tex.stackexchange.com/questions/54368/tikz-nodes-centering-with-small-font
\tikzset{%
	%every node/.append style={execute at begin node=\footnotesize}
	%every node/.style={font=\footnotesize},
	every picture/.style={font issue=\footnotesize},
	font issue/.style={execute at begin picture={#1\selectfont}},
}

\tikzset{%
	every fit/.style={text badly centered},
	% http://tex.stackexchange.com/questions/193052/rotate-only-the-shape
	hexa/.style={shape border rotate=90,draw,shape=regular polygon,regular polygon sides=6,minimum size=\hexasize,inner sep=0pt,outer sep=0pt},
	>=stealth',
	%shorten >=1pt,
	%textdsc/.style={font=\tiny,fill=white,inner xsep=0pt,inner ysep=2pt},
	textdsc/.style={font=\tiny\smaller},
	ue/.style={draw,circle,double,text width=2*\raio,inner sep=0pt},
	bs/.style={draw,circle,fill=black,text width=2*\raio,inner sep=0pt},
	%pcentro/.style={fill=black},
}

% Use this or 'standalone' class with options as above
\usepackage[active,tightpage]{preview}
\PreviewEnvironment{tikzpicture}

% Some useful measurements
\newlength{\twoside}
\setlength{\twoside}{\dimexpr0.5\textwidth-0.5\columnsep\relax}

\begin{document}

\begin{tikzpicture}

	%\node[rectangle,draw,minimum width=\twoside] (frame) {};		

	% Layout
	\node[hexa] (centro) at (0,0) {};

%	% D2D-capable MSs
	\coordinate (d2due1pos) at ($(centro.center)!0.6!(centro.270)$);
	\node[ue] (d2due1) at (d2due1pos) {};
	
	\coordinate (d2due2pos) at ($(centro.center)!0.9!(centro.330)$);
	\node[ue] (d2due2) at (d2due2pos) {};
	
	\coordinate (d2due3pos) at ($(centro.center)!0.3!(centro.360)$);
	\node[ue] (d2due3) at (d2due3pos) {};
	
	\coordinate (d2due3pos) at ($(centro.center)!0.5!(centro.140)$);
	\node[ue] (d2due3) at (d2due3pos) {};
	
	\coordinate (d2due3pos) at ($(centro.center)!0.8!(centro.80)$);
	\node[ue] (d2due3) at (d2due3pos) {};
		
	\coordinate (d2due3pos) at ($(centro.center)!0.15!(centro.200)$);
	\node[ue] (d2due3) at (d2due3pos) {};

	\newcommand*{\List}{centro}
	\getNoOfElements{\List}{\No}
	\foreach \x [count=\xi] in \List {%
		\node[bs] (\xi) at (\x) {};
	}

	% Legend
  \matrix (legend) [%
		draw=black,
		cells={anchor=center},
		inner xsep=3pt,inner ysep=2pt, % same as in legend of plots
		below=\lstart of centro.south,
		anchor=north,
		matrix of nodes,
		ampersand replacement=\&,
		nodes={outer sep=2pt,text depth=0.15em,anchor=west}, % 
  ]
  {%
		|[bs]|{} \& {BS} \\
		|[ue]|{} \& {Dispositivos M\'oveis} \\
  };

	
\end{tikzpicture}

\end{document}
