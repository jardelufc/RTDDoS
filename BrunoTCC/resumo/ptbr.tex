% resumo em português
\setlength{\absparsep}{18pt} % ajusta o espaçamento dos parágrafos do resumo
\begin{resumo}
      Ataques Distribuídos de Negação de Serviço (DDoS) figuram uma das principais ameaças a redes de computadores atualmente. Sistemas de Detecção de Intrusão (IDS) são responsáveis por detectarem pacotes maliciosos e proverem medidas cabíveis para que a ameaça não cause maiores danos ao servidor/\textit{host}. O presente trabalho visa o estudo, implementação e validação de um \textit{framework} de detecção de ataques DDoS publicado este ano em um periódico cientificamente reconhecido. O \textit{framework} foi testado e validado, utilizando-se duas bases de dados (\textit{Data Mining} e DARPA), no qual obteve alto desempenho medido em termos de taxa de acertos.
      
\textbf{Palavras-chaves}: DDoS. Segurança em redes. Tempo real.
\end{resumo}