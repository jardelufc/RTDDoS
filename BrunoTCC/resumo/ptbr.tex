% resumo em português
\setlength{\absparsep}{18pt} % ajusta o espaçamento dos parágrafos do resumo
\begin{resumo}

A demanda crescente dos usuários de redes móveis por novos serviços que necessitam de altas taxas de transmissão e níveis aceitáveis de Qualidade de Serviço (QoS, do inglês \textit{Quality of Service}) aumentou a importância do estudo de algoritmos de alocação de recursos de rádio. A eficiência desses algoritmos se faz necessária para que possa existir uma justa alocação dos recursos entre os usuários e que os requerimentos de cada usuário sejam atendidos. Neste contexto, este trabalho propõe-se a fazer um estudo bibliográfico sobre algoritmos de alocação de recursos de rádio presentes na literatura para cenários com serviço único ou múltiplos serviços para o enlace direto de redes celulares da 4ª Geração (4G). Com base neste estudo, alguns algoritmos são selecionados para que seus desempenhos sejam analisados com base na métrica de satisfação do usuário. Para realização desta análise, foi feita uma modelagem e implementação de um ambiente de simulação para sistemas celulares 4G que seguem o padrão LTE (do inglês \textit{Long Term Evolution}), os quais são baseados em Múltiplo Acesso por Divisão de Frequências Ortogonais (OFDMA, do inglês \textit{Orthogonal Frequency Division Multiple Access}). O cenário de simulação utilizado neste trabalho é composto por dois serviços: o serviço de Voz sobre IP (VoIP, do inglês \textit{Voice over IP}), que é um serviço de Tempo Real (RT, do inglês \textit{Real Time}), e um serviço com Taxa de Bits Constante (CBR, do inglês \textit{Constant Bit Rate}), que nós assumimos como um serviço de Tempo Não Real (NRT, do inglês \textit{Non Real Time}). Após as simulações neste cenário, alguns gráficos são construídos ilustrando a porcentagem de usuários satisfeitos de acordo com a quantidade de usuários presentes no sistema. Os resultados mostram que o algoritmo que apresenta o melhor desempenho é aquele que aloca os recursos para uma determinada classe de serviço com base na principal métrica de QoS daquele serviço, ou seja, a métrica que define se os usuários daquele serviço têm seus requerimentos atendidos e, por consequência, estão satisfeitos.
      
\textbf{Palavras-chaves}: Algoritmos de Alocação de Recursos de Rádio. Qualidade de Serviço. Redes Celulares de 4G. Satisfação do Usuário.
\end{resumo}