% resumo em inglês
\begin{resumo}[Abstract]
 \begin{otherlanguage*}{english}
 	
   The increasing demand of mobile networks users for new services that require high transmission rates and acceptable levels of Quality of Service (QoS) has raised the importance of the study of radio resource allocation algorithms. The efficiency of such algorithms is essential so that there exists a fair resource allocation among users and that the requirements of each individual user are met. In this context, the purpose of this work is to perform a bibliographic study about radio resource allocation algorithms present in the literature for single- or multi-service scenarios in the downlink of 4th Generation (4G) cellular networks. On the basis of this study, some algorithms are selected to be implemented and have their performance evaluated based on the user satisfaction metric. In order to perform this evaluation, we have modeled and implemented a simulation environment for 4G cellular systems that follow the Long Term Evolution (LTE) standard, which are based on Orthogonal Frequency Division Multiple Access (OFDMA). The simulation scenario employed in this work is comprised of two services: the Voice over IP (VoIP) service, which is a Real Time (RT) service, and a Constant Bit Rate (CBR) service, which we assume as a Non Real Time (NRT) service. After simulations in such scenario, some results were generated illustrating the percentage of satisfied users according to the amount of users present in the system. The results show that the algorithm that presents the best performance is the one that allocates the resources for a certain service class based on the main QoS metric of that service, i.e., the metric that defines if the users of that service have their requirements met and, as a consequence, are satisfied.
 
   \noindent 
   \textbf{Key-words}: Radio Resource Allocation Algorithms. Quality of Service. 4G Cellular Networks. User Satisfaction.
 \end{otherlanguage*}
\end{resumo}