\chapter[Metodologia]{Metodologia}
\label{Simulacao}

Neste capítulo, os detalhes do \textit{fremework} serão apresentados bem como a modelagem do ambiente de simulação utilizado para a detecção de Ataques [DDOS]
% ----------------------------------------------------------

\section{Modelo de correlação NaHiD}

Neste trabalho, o \textit{framework} utilizado baseia-se na correlação proposta por \cite{NHOQUE} chamada NaHiD, cujo objetivo é distinguir objetos de tráfego normais e maliciosos. Tal medida leva em consideração principalmente desvio padrão e média de cada objeto, ponderando cada elemento como mostrado na fórmula a seguir:  

\begin{equation}
	NaHiD(X,Y) = 1 - \frac{1}{n} \sum_{i=1}^{n} \frac{\left(|X(i) - Y(i)|\right)}{||\mu{X} - sX| - X(i)| + ||\mu{Y} - sY| - Y(i)|}
\end{equation}
onde
\begin{itemize}
	\item $\mu{X}$: Média do objeto de tráfego X.
 	\item $\mu{X}$: Média do objeto de tráfego Y.
	\item $sX$: Desvio padrão do objeto de tráfego X.
	\item $sY$: Desvio Padrão do objeto de tráfego Y.
\end{itemize}
As provas de simetria e identidade da correlação podem ser encontradas em \cite{NHOQUE}

\section{Detecção de Ataques DDOS usando NaHiD}
\label{Sec:NaHiD_VERC}
Para a avaliação do trabalho, duas bases de tráfegos de rede foram escolhidas: DARPA e  DataMining[escolher melhor esse nome] os quais serão mais detalhados aa seguir 
\subsection{DARPA - MIT}
 A base de dados DARPA foi produzida por pesquisadores do \textit{Lincoln Laboratory} do Instituto de Tecnologia de Massachusetts nos Estados Unidos e tem por objetivo coletar dados de tráfego de rede da Força Aérea do país para encontrar vulnerabilidades em seu sistema bem como ser utilizado para avaliações futuras . Os dados foram coletados e passaram por uma fase de treinamento de 7 semanas com 38 tipos de ataques para simular ameaças internas a rede. O ambiente de rede era composto por duas partes: a rede interna da Força aérea e a rede externa que representava a Internet; ambos conectados por meio de um roteador como mostra a figura abaixo
 
 	\begin{tikzpicture}
 	\path [
 	mindmap,
 	text = white,
 	level 1 concept/.append style =
 	{font=\small\bfseries, sibling angle=180},
 	level 2 concept/.append style =
 	{font=\tiny\bfseries},
 	level 3 concept/.append style =
 	{font=\tiny\bfseries},
 	tex/.style     = {concept, ball color=blue,
 		font=\normalsize\bfseries},
 	engines/.style = {concept, ball color=green!50!black},
 	formats/.style = {concept, ball color=blue!50!black},
 	systems/.style = {concept, ball color=red!90!black},
 	]
 	node [tex] {Roteador} [clockwise from=180]
 	child[concept color=green!50!black, nodes={engines}] {
 		node {Base Aérea} [clockwise from=90]}
 	child [concept color=red, nodes={systems}] {
 		node {Internet} [clockwise from=90]};
 	\end{tikzpicture}
\\
Tal banco de dados é disponibilizado pela DARPA em um arquivo de extensão tcpdump, bem como a listagem de tráfegos normais e ataques rotulados como mostrados nas tabelas seguir

\begin{table}[!b]
	\centering
	\begin{threeparttable}
		\caption{Exemplo base de dados DARPA}
		\label{Tab:WiresharkEx}
		%	\small
		\begin{tabular}{c c c c c c}
			\toprule
			\textbf{Número} & \textbf{Tempo} & \textbf{Origem} & \textbf{Destino}  & \textbf{Protocolo} & \textbf{Tamanho}[bytes]
			\\ \midrule
			1 &  18:56:12.1386 &  192.168.0.20 & 192.168.0.30 & TCP & 60  \\ \midrule
			2 &  18:56:12.1391 & 192.168.0.30 & 192.168.0.20 & TCP & 60  \\ \midrule
			3 &  18:56:12.1588 & 192.168.0.30 & 192.168.0.20 & TELNET & 84  \\ \midrule
			4 &  18:56:12.2099 &  192.168.0.20 & 192.168.0.30 & TCP & 60  \\ \midrule
			5 &  18:56:13.0567 &  192.168.0.20 & 192.168.0.30 & TELNET & 69    \\ \midrule
			6 &  18:56:13.0584 & 192.168.0.30 & 192.168.0.20 & TELNET & 66   \\ \midrule
			7 &  18:56:13.0626 &  192.168.0.20 & 192.168.0.30 & TELNET & 72  \\ \midrule
			8 & 18:56:13.0821 & 192.168.0.30 & 192.168.0.20 & TCP & 60  \\ \bottomrule
		\end{tabular}
		{Fonte: Elaborada pelo autor, baseada em \cite{DARPA}.}
	\end{threeparttable}
\end{table}

No presente trabalho ferramentas como edicap e tcpdump foram utilizadas para o tratamento desse dataset. Assim, algumas considerações devem ser feitas:
\begin{itemize}
	\item Janela de um segundo de tráfego
	\item Cálculo de entropia, variação de IPs origem e taxa de pacotes média
	\item Cálculo da correlação NaHiD
\end{itemize}
.
\subsection{DataMining}
Outra base de dados estudada no trabalho foi a desenvolvida por \cite{DataMining} a qual consta em sua totalidade por ataques DDOS de quatro tipos:
\begin{itemize}
	\item SIDDOS
	\item HTTP Flood
	\item UDDP Flood
	\item Smurf
\end{itemize}

A Tabela abaixo mostra os campos do \textit{dataset} 

\begin{table}[!b]
	\centering
	\begin{threeparttable}
		\caption{Exemplo base de dados DARPA}
		\label{Tab:DataMining}
		%	\small
		\begin{tabular}{c c }
			\toprule
			\textbf{Número} & \textbf{Tempo}
			\\ \midrule
			1 &  Endereço IP origem  \\ \midrule
			2 &  Endereço IP destino  \\ \midrule
			3 &  Id do pacote  \\ \midrule
			4 &  Nó origem  \\ \midrule
			5 &  Nó destino  \\ \midrule
			6 &  Tipo de pacote  \\ \midrule
			7 &  Tamanho do pacote  \\ \midrule
			8 &  Flags  \\ \midrule
			9 &   Id da flag  \\ \midrule
			10 &  Número de sequência  \\ \midrule
			11 &  Número de pacotes  \\ \midrule
			12 &  Número de bytes  \\ \midrule
			13 &  Nome do nó origem  \\ \midrule
			14 &  Nome do nó destino  \\ \midrule
			15 &  Entrada de pacote  \\ \midrule
			16 &  Saída de pacote  \\ \midrule
			17 &  Taxa de pacotes Recebidos \\ \midrule%%%%%%%%%%%%%%%%%%
			18 &  Atraso de nó do pacote  \\ \midrule
			19 &  Taxa de pacotes\\ \midrule
			20 &  Taxa de bytes  \\ \midrule
			21 &  Tamanho  médio do pacote  \\ \midrule
			22 &  Utilização  \\ \midrule
			23 &  Atraso de pacote  \\ \midrule
			24 &  Tempo de envio do pacote  \\ \midrule
			25 &  Tempo de pacote reservado  \\ \midrule
			26 &  Primeiro pacote enviado  \\ \midrule
			27 &  Último pacote reservado \\ \bottomrule
		\end{tabular}
		{Fonte: Elaborada pelo autor, baseada em \cite{DataMining}.}
	\end{threeparttable}
\end{table}

Algumas considerações foram tomadas para a análise dessa base de dados:
\begin{itemize}
	\item Para construir a janela de um segundo, considerou-se a soma de todos os atrasos por pacote:
	\begin{itemize}
	 \item Atraso de nó do pacote.
	 \item  Atraso de pacote.
	 \item Tempo de pacote reservado.
	\end{itemize}
	\item A média das taxas dos pacotes foi considerada dentro da janela de um segundo.
	\item Por ser um dataset composto apenas por ataques, a comparação com o limiar inverte-se para denotar o quanto dois pacotes são parecidos na correlação.
\end{itemize}