% resumo em português
\setlength{\absparsep}{18pt} % ajusta o espaçamento dos parágrafos do resumo
\begin{resumo}
\noindent 
\textbf{Palavras Chaves}: Módulo,FPGA, Segurança, Tempo Real, Correlação, Hardware.

Sistemas de detecção de intrusão são cada vez mais necessários para garantir a segurança de  serviços na internet, uma vez que as ameaças na rede vem desenvolvendo-se cada vez mais. Ataques do tipo DDoS são muito comuns atualmente, uma vez que já existem recursos suficientes para realizar esse tipo de ataque em tempo real. Apesar de existirem soluções em softwares para detectar ataques DDoS, muitas são ineficientes. Nesse trabalho foi implementado um \textit{soft ip core} para \textit{FPGAs} que realiza a detecção de ataque DDoS em tempo real, em um tempo de menos de 1$\mu$s. Além disso, o módulo implementado  alia baixa utilização de recursos, pois faz uso de aritmética de ponto fixo, com uma elevada precisão quando comparado a implementação em software com ponto flutuante. O módulo foi implementado em nível RTL e sintetizado em uma \textit{FPGA Artix} da Série 7 da \textit{Xilinx}. 

\end{resumo}