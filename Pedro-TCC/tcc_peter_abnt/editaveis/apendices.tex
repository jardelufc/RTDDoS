\chapter{Códigos do Módulo}
\label{Apx:A}

\definecolor{vgreen}{RGB}{104,180,104}
\definecolor{vblue}{RGB}{49,49,255}
\definecolor{vorange}{RGB}{255,143,102}
\lstdefinestyle{verilog-style}
{
	language=Verilog,
	basicstyle=\small\ttfamily,
	keywordstyle=\color{vblue},
	identifierstyle=\color{black},
	commentstyle=\color{vgreen},
	numbers=left,
	numberstyle=\tiny\color{black},
	numbersep=10pt,
	tabsize=8,
	moredelim=*[s][\colorIndex]{[}{]},
	literate=*{:}{:}1	
}

\makeatletter
\newcommand*\@lbracket{[}
\newcommand*\@rbracket{]}
\newcommand*\@colon{:}
\newcommand*\colorIndex{%
	\edef\@temp{\the\lst@token}%
	\ifx\@temp\@lbracket \color{black}%
	\else\ifx\@temp\@rbracket \color{black}%
	\else\ifx\@temp\@colon \color{black}%
	\else \color{vorange}%
	\fi\fi\fi
}
\makeatother

\section{Código Nahid}
\label{nahidcode}
\begin{lstlisting}[style={verilog-style}]
module Nahid(clk,reset,Rx1,Rx2,Rx3,Ry1,Ry2,Ry3,R11resul,threshold);
//entrada do controller
input clk,reset;
//entrada do datapath
input [10:0] Rx1,Rx2,Rx3,Ry1,Ry2,Ry3;
output[19:0] R11resul;
input  [19:0] threshold;
reg selmul3_1,selmul3_2,seladd5_2,seladd5_1,selr7,selsqrt,clr1,clr2,
clr3,clr4,clr5,clr6,clr7,clr8,clr9,clr10,clr11,enable1,enable2,
enable3,enable4,enable5,enable6,enable7,enable8,enable9,enable10,
enable11,valid,insqrt,valid2,inbottom,intop;
reg clrx1,clrx2,clrx3,clry1,clry2,clry3,enablex1,enablex2,enablex3,
enabley1,enabley2,enabley3;
reg [1:0] selmul2_2,selmul2_1,selmul1_2,selmul1_1,seladd2_2,seladd1_1,
seldiv_2,seldiv_1,selr11,selr10,selr9,selr8,selr6,selr4,selr3,
selr2,selr1;
reg [2:0] seladd4_2,seladd4_1,seladd3_2,seladd3_1,seladd2_1,seladd1_2;
reg [1:0] opadd1,opadd2,opadd3,opadd4,opadd5;
reg [19:0] R11resul;
reg attack,finish;
Datapath data1 (.Rx1(Rx1),.Rx2(Rx2),.Rx3(Rx3),.Ry1(Ry1),.Ry2(Ry2),
.Ry3(Ry3),.selmul3_1(selmul3_1),.selmul3_2(selmul3_2),
.seladd5_2(seladd5_2),.seladd5_1(seladd5_1),.selr7(selr7),
.selsqrt(selsqrt),.selmul2_2(selmul2_2),.selmul2_1(selmul2_1),
.selmul1_2(selmul1_2),.selmul1_1(selmul1_1),.seladd2_2(seladd2_2),
.seladd1_1(seladd1_1),.seldiv_2(seldiv_2),.seldiv_1(seldiv_1),
.selr11(selr11),.selr10(selr10),.selr9(selr9),.selr8(selr8),
.selr6(selr6),.selr4(selr4),.selr3(selr3),.selr2(selr2),
.selr1(selr1),.seladd4_2(seladd4_2),.seladd4_1(seladd4_1),
.seladd3_2(seladd3_2),.seladd3_1(seladd3_1),.seladd2_1(seladd2_1),
.seladd1_2(seladd1_2),.clk(clk),.clr1(clr1),.clr2(clr2),
.clr3(clr3),.clr4(clr4),.clr5(clr5),.clr6(clr6),.clr7(clr7)
,.clr8(clr8),.clr9(clr9),.clr10(clr10),.clr11(clr11),
.enable1(enable1),.enable2(enable2),.enable3(enable3),
.enable4(enable4),.enable5(enable5),.enable6(enable6),
.enable7(enable7),.enable8(enable8),.enable9(enable9),
.enable10(enable10),.enable11(enable11), .opadd1(opadd1),.opadd2(opadd2),
.opadd3(opadd3),.opadd4(opadd4),.opadd5(opadd5),
.clrx1(clrx1),.clrx2(clrx2),.clrx3(clrx3),
.clry1(clry1),.clry2(clry2),
.clry3(clry3),.enablex1(enablex1),.enablex2(enablex2),
.enablex3(enablex3),.enabley1(enabley1),.enabley2(enabley2),
.enabley3(enabley3),.insqrt(insqrt),.valid(valid),
.valid2(valid2),.inbottom(inbottom),
.intop(intop),.R11resul(R11resul));

controller c1 (.clk(clk),.reset(reset),.selmul3_1(selmul3_1),
.selmul3_2(selmul3_2),.seladd5_2(seladd5_2),.seladd5_1(seladd5_1),
.selr7(selr7),.selsqrt(selsqrt),.selmul2_2(selmul2_2),
.selmul2_1(selmul2_1),.selmul1_2(selmul1_2),.selmul1_1(selmul1_1),
.seladd2_2(seladd2_2),.seladd1_1(seladd1_1),.seldiv_2(seldiv_2),
.seldiv_1(seldiv_1),.selr11(selr11),.selr10(selr10),.selr9(selr9),
.selr8(selr8),.selr6(selr6),.selr4(selr4),.selr3(selr3),
.selr2(selr2),.selr1(selr1),.seladd4_2(seladd4_2),
.seladd4_1(seladd4_1),.seladd3_2(seladd3_2),
.seladd3_1(seladd3_1),.seladd2_1(seladd2_1),
.seladd1_2(seladd1_2),.clr1(clr1),.clr2(clr2),
.clr3(clr3),.clr4(clr4),.clr5(clr5),.clr6(clr6),
.clr7(clr7),.clr8(clr8),.clr9(clr9),.clr10(clr10),
.clr11(clr11),.enable1(enable1),.enable2(enable2),
.enable3(enable3),.enable4(enable4),.enable5(enable5),
.enable6(enable6),.enable7(enable7),.enable8(enable8),
.enable9(enable9),.enable10(enable10),.enable11(enable11),
.opadd1(opadd1),.opadd2(opadd2),.opadd3(opadd3),.opadd4(opadd4)
,.opadd5(opadd5),.clrx1(clrx1),.clrx2(clrx2),.clrx3(clrx3),
.clry1(clry1),.clry2(clry2),.clry3(clry3),.enablex1(enablex1),
.enablex2(enablex2),.enablex3(enablex3),.enabley1(enabley1),
.enabley2(enabley2),.enabley3(enabley3),
.insqrt(insqrt),.valid(valid),.valid2(valid2),.inbottom(inbottom),
.intop(intop),.finish(finish));

always_comb
	begin
		begin
			attack=0;
				if (finish)
					if(threshold>R11resul)
						attack=1;
					else
						attack=0;
					end
				end
endmodule
\end{lstlisting}

\section{Código Datapath}
\label{codedatapath}

\begin{lstlisting}[style={verilog-style}]

module Datapath(Rx1,Rx2,Rx3,Ry1,Ry2,Ry3,
selmul3_1,selmul3_2,seladd5_2,seladd5_1,selr7,selsqrt,selmul2_2
,selmul2_1,selmul1_2,selmul1_1,seladd2_2,seladd1_1,seldiv_2
,seldiv_1,selr11,selr10,selr9,selr8,selr6,selr4,selr3,selr2
,selr1,seladd4_2,seladd4_1,seladd3_2,seladd3_1,seladd2_1
,seladd1_2,clk,clr1,clr2,clr3,clr4,clr5,clr6,clr7,clr8,clr9
,clr10,clr11,clrx1,clrx2,clrx3,clry1,clry2,clry3,enable1
,enable2,enable3,enable4,enable5,enable6,enable7,enable8
,enable9,enable10,enable11,enablex1,enablex2,enablex3,enabley1
,enabley2,enabley3,datasqrt,datadiv,datar11,datar10,datar9
,datar8,datar7,datar6,datar5,datar4,datar3,datar2,datar1
,opadd1,opadd2,opadd3,opadd4,opadd5,R11resul
,valid,insqrt,valid2,inbottom,intop);
input [10:0] Rx1,Rx2,Rx3,Ry1,Ry2,Ry3;
//Seletores
input  selmul3_1,selmul3_2,seladd5_2,seladd5_1,selr7,selsqrt,clk,clr1
,clr2,clr3,clr4,clr5,clr6,clr7,clr8,clr9,clr10,clr11,clrx1
,clrx2,clrx3,clry1,clry2,clry3,enable1,enable2,enable3,enable4
,enable5,enable6,enable7,enable8,enable9,enable10,enable11
,enablex1,enablex2,enablex3,enabley1,enabley2,enabley3,datasqrt
,datadiv,datar11,datar10,datar9,datar8,datar7,datar6,datar5
,datar4,datar3,datar2,datar1,insqrt,inbottom,intop;
input [1:0] selmul2_2,seladd4_1,selmul2_1,selmul1_2,selmul1_1,seladd2_2
,seladd1_1,seldiv_2,seldiv_1,selr11,selr10,selr9,selr8
,selr6,selr4,selr3,selr2,selr1,opadd1,opadd2,opadd3,opadd4,opadd5;
input [2:0] seladd4_2,seladd3_2,seladd3_1,seladd2_1,seladd1_2; 
output valid,valid2;
output [19:0] R11resul ;
wire  clrx1,clrx2,clrx3,clry1,clry2,clry3,enablex1,enablex2,enablex3
,enabley1,enabley2,enabley3,valid,valid2;
wire [23:0] op2add5,op1add5,op2add4,op1add4,op2add3,op1add3
,op2add2,op1add2,op2add1,op1add1;
wire [23:0] add5resul,add4resul,add3resul,add2resul,add1resul,add5shifted
,add4shifted,add3shifted,add2shifted,add1shifted;
wire [23:0] mul3resul,mul2resul,mul1resul,
op2div,op1div,sqrtresul,
div;
wire [10:0] op2mul3,op1mul3,op2mul2,op1mul2,op2mul1,op1mul1,x1r,x2r,x3r
,y1r,y2r,y3r;
wire [22:0] Rx1new,Rx2new,Rx3new,Ry1new,Ry2new,Ry3new,R7ex,R5ex
,R11ex,divnew;
wire [19:0] divresul;
wire [23:0] opR1,opR2,opR3,opR4,opR5,opR6,opR7,opR8,opR9,opR10,opR11,
opsqrt,sqrtresulnew;
wire [23:0] R10resul,R9resul,R8resul,R7resul,R6resul,R5resul,R4resul,
R3resul,R2resul,R1resul;
wire [10:0] R7resulnew,R1resulnew;
wire [11:0] op1divnew,op2divnew;
//extends (Modulo para ajustar entradas de 11 para 23 bits)
extend ex1 (.a(x1r),.y(Rx1new));
extend ex2 (.a(x2r),.y(Rx2new));
extend ex3 (.a(x3r),.y(Rx3new));
extend ey1 (.a(y1r),.y(Ry1new));
extend ey2 (.a(y2r),.y(Ry2new));
extend ey3 (.a(y3r),.y(Ry3new));
extend #(12) er7 (.a(R7resul),.y(R7ex));
extend #(16) sqrt (.a(sqrtresul),.y(sqrtresulnew));
extend er5 (.a(R5resul),.y(R5ex));
extend er1  (.a(R1resul),.y(R1ex));
extend #(20) er11 (.a(R11resul),.y(R11ex));
extend #(20) ediv (.a(divresul),.y(divnew));
//Reduce(Modulo para ajustar Saidas de 24 para 12 bits)

reduce rr7 (.a(R7resul),.y(R7resulnew));
reduce rr1 (.a(R1resul),.y(R1resulnew));


reduce rop1 (.a(op1div),.y(op1divnew));
reduce rop2 (.a(op2div),.y(op2divnew));



/*Multiplexadores parte 1 
Primeira coluna da arquitetura
-- O que recebe?
Entradas dos dados e a 
realimentacao de registradores e variaveis,
 alem disso coloca as entradas em nos valores dos adders.
*/

//Mux dos multiplicadores 
//Multiplexadores do Multiplicador 3 
mux2 mux2mul3(.a(y3r),.b(x3r),.sel(selmul3_2),.y(op2mul3));
mux2 mux1mul3(.a(y3r),.b(x3r),.sel(selmul3_1),.y(op1mul3));
//Multiplexadores do Multiplicador 2
mux4 mux2mul2(.a(R7resulnew),.b(y2r),.c(x2r),.d(0),.sel(selmul2_2)
,.y(op2mul2));
mux4 mux1mul2(.a(R7resulnew),.b(y2r),.c(x2r),.d(0),.sel(selmul2_1)
,.y(op1mul2));
//Multiplexadores do Multiplicador 1
mux4 mux2mul1(.a(R1resulnew),.b(y1r),.c(x1r),.d(0),.sel(selmul1_2)
,.y(op2mul1));
mux4 mux1mul1(.a(R1resulnew),.b(y1r),.c(x1r),.d(0),.sel(selmul1_1)
,.y(op1mul1));



//Mux dos Somadores 

//Multiplexadores do Somador 5
mux2 mux2add5(.a(R11ex),.b(divnew),.sel(seladd5_2),.y(op2add5));
mux2 mux1add5(.a(24'b000000000000000100000000),.b(R11ex)
,.sel(seladd5_1),.y(op1add5));
//Multiplexadores do Somador 4
mux6 mux2add4(.a(R1resul),.b(R2resul),.c(Ry3new),.d(Ry2new)
,.e(Ry1new),.f(0),.g(0),.h(0),.sel(seladd4_2),.y(op2add4));
mux4 mux1add4(.a(R8resul),.b(Rx3new),.c(Rx2new),.d(Rx1new)
,.sel(seladd4_1),.y(op1add4));
//Multiplexadores do Somador 3
mux6 mux2add3(.a(R10resul),.b(Ry2new),.c(Rx2new),.d(R9resul)
,.e(R4resul),.f(R7ex),.g(0),.h(0),.sel(seladd3_2),.y(op2add3));
mux6 mux1add3(.a(R8resul),.b(R2resul),.c(R10resul),.d(R6resul)
,.e(Ry3new),.f(0),.g(0),.h(0),.sel(seladd3_1),.y(op1add3));
//Multiplexadores do Somador 2
mux4 mux2add2(.a(R6resul),.b(R9resul),.c(R3resul),.d(Ry2new)
,.sel(seladd2_2),.y(op2add2));
mux6 mux1add2(.a(R7ex),.b(R1resul),.c(R8resul),.d(R2resul)
,.e(Ry1new),.f(0),.g(0),.h(0),.sel(seladd2_1),.y(op1add2));
//Multiplexadores do Somador 1
mux6 mux2add1(.a(Ry1new),.b(R10resul),.c(R4resul),.d(Rx3new)
,.e(Rx1new),.f(0),.g(0),.h(0),.sel(seladd1_2),.y(op2add1));
mux4 mux1add1(.a(R8resul),.b(R2resul),.c(R1resul),.d(Rx2new)
,.sel(seladd1_1),.y(op1add1));

/*Multiplicadores e Somadores
segunda coluna da arquitetura
-- O que recebe?
Os dados que os multiplexadores parte 1 selecionam.
*/
mul mul3 (.a(op1mul3),.b(op2mul3),.mul(mul3resul));
mul mul2 (.a(op1mul2),.b(op2mul2),.mul(mul2resul));
mul mul1 (.a(op1mul1),.b(op2mul1),.mul(mul1resul));

adder add5(.a(op1add5),.b(op2add5),.op(opadd5),.o(add5resul));
adder add4(.a(op1add4),.b(op2add4),.op(opadd4),.o(add4resul));
adder add3(.a(op1add3),.b(op2add3),.op(opadd3),.o(add3resul));
adder add2(.a(op1add2),.b(op2add2),.op(opadd2),.o(add2resul));
adder add1(.a(op1add1),.b(op2add1),.op(opadd1),.o(add1resul));

/*Multiplexadores parte 2 
terceira coluna da arquitetura
-- O que recebe?
Saida dos modulos da parte 2 e variaveis.
*/

//mux raiz quadrada
mux2r muxsqrt(.a(R9resul),.b(R3resul),.sel(selsqrt),.y(opsqrt));
//mux divisor
mux4r mux2div(.a(R10resul),.b(R6resul),.c(R1resul),.sel(seldiv_2)
,.y(op2div));
mux4r mux1div(.a(R3resul),.b(R4resul),.c(R5ex)
,.sel(seldiv_1),.y(op1div));
//mux dos registradores
mux4r muxr11(.a(add5resul),.b(add5resul),.c(divresul),.sel(selr11)
,.y(opR11));

mux4r muxr10(.a(add3resul),.b(add4resul),.c(add1resul),.d(mul3resul)
,.sel(selr10),.y(opR10));

mux4r muxr9(.a(add3resul),.b(sqrtresulnew),.c(mul2resul),.sel(selr9)
,.y(opR9));

mux4r muxr8(.a(add1resul),.b(add2resul),.c(mul1resul),.sel(selr8)
,.y(opR8));

mux2r muxr7(.a(add3resul),.b(add2resul),.sel(selr7),.y(opR7));

mux4r muxr6(.a(sqrtresulnew),.b(add4resul),.c(add1resul),.d(add2resul)
,.sel(selr6),.y(opR6));

mux4r muxr4(.a(add4resul),.b(mul1resul),.c(mul3resul),.sel(selr4)
,.y(opR4));

mux4r muxr3(.a(add4resul),.b(add3resul),.c(mul2resul),.d(0)
,.sel(selr3),.y(opR3));

mux4r muxr2(.a(add2resul),.b(add3resul),.c(mul1resul),.sel(selr2)
,.y(opR2));

mux4r muxr1(.a(add4resul),.b(add1shifted),.c(add1resul),.sel(selr1)
,.y(opR1));

/*Registradores, div e sqrt
quarta coluna da arquitetura
-- O que recebe?
Os dados que os multiplexadores parte 3 selecionam.
*/  
//div div1 (.a(op1divnew),.b(op2divnew),.div(divresul));

div_gen_1 divider (clk,inbottom,op2div,intop,op1div,valid2,divresul);

//sqrt sqrt1(.b(opsqrt),.square(sqrtresul));

cordic_1 sqrtcalc (clk,insqrt,opsqrt,valid,sqrtresul);


register r1 (.in(opR1),.clk(clk),.clr(clr1)
,.enable(enable1),.o(R1resul));
register r2 (.in(opR2),.clk(clk),.clr(clr2)
,.enable(enable2),.o(R2resul));
register r3 (.in(opR3),.clk(clk),.clr(clr3)
,.enable(enable3),.o(R3resul));
register r4 (.in(opR4),.clk(clk),.clr(clr4)
,.enable(enable4),.o(R4resul));
register #(11) r5  (.in(add4resul),.clk(clk),.clr(clr5)
,.enable(enable5),.o(R5resul));
register r6 (.in(opR6),.clk(clk),.clr(clr6)
,.enable(enable6),.o(R6resul));
register #(12) r7 (.in(opR7),.clk(clk),.clr(clr7)
,.enable(enable7),.o(R7resul));
register r8 (.in(opR8),.clk(clk),.clr(clr8)
,.enable(enable8),.o(R8resul));
register r9 (.in(opR9),.clk(clk),.clr(clr9)
,.enable(enable9),.o(R9resul));
register r10 (.in(opR10),.clk(clk),.clr(clr10)
,.enable(enable10),.o(R10resul));
register #(20) r11 (.in(opR11),.clk(clk),.clr(clr11)
,.enable(enable11),.o(R11resul));

register #(11) x1 (.in(Rx1),.clk(clk),.clr(clrx1)
,.enable(enablex1),.o(x1r));
register #(11) x2 (.in(Rx2),.clk(clk),.clr(clrx2)
,.enable(enablex2),.o(x2r));
register #(11) x3 (.in(Rx3),.clk(clk),.clr(clrx3)
,.enable(enablex3),.o(x3r));
register #(11) y1 (.in(Ry1),.clk(clk),.clr(clry1)
,.enable(enabley1),.o(y1r));
register #(11) y2 (.in(Ry2),.clk(clk),.clr(clry2)
,.enable(enabley2),.o(y2r));
register #(11) y3 (.in(Ry3),.clk(clk),.clr(clry3)
,.enable(enabley3),.o(y3r));
endmodule


\end{lstlisting}

\section{Código Extend}

\label{codeextend}
\begin{lstlisting}[style={verilog-style}]
module extend #(parameter WIDTH=11) (input [WIDTH-1:0] a
,output reg [23:0] y);

	always_comb

		y = {0, a};   

endmodule
\end{lstlisting}

\section{Código Reduce}
\label{codereduce}
\begin{lstlisting}[style={verilog-style}]
module reduce(input [23:0] a,
output reg [10:0] y);

	always_comb

		y =a[10:0];  

endmodule

\end{lstlisting}

\section{Código Mux}
\label{codemux}
\begin{lstlisting}[style={verilog-style}]
module mux2(input [22:0] a,b,
input  sel, 
output reg [22:0] y );
	always_comb 
 		begin                 
			case(sel)    
				0: y =a;
				1: y =b;
				default: y=23'bx;
			endcase
		end

endmodule

\end{lstlisting}

\section{Código Mul}
\label{codemul}
\begin{lstlisting}[style={verilog-style}]
module mul(a,b,mul);
input [10:0] a,b;
output [23:0]mul;

	assign mul=a*b;

endmodule
\end{lstlisting}

\section{Código Adder}
\label{codeadder}
\begin{lstlisting}[style={verilog-style}]
module adder #(parameter WIDTH=23)
(input  [WIDTH-1:0] a,b,
input [1:0]op,
output  reg [WIDTH:0] o);
	always_comb
		begin
			case (op)
				2'b00:o = a+b;
				2'b01:o = (a+b)>>2;
				2'b10: 
				 if(a>b)
					begin
						o=a-b;
					end
				else
						o=b-a;
				2'b11:o=$unsigned(b-a)>>2;
				default:o=24'bx;
			endcase
		end
endmodule

\end{lstlisting}

\section{Código Register}
\label{coderegister}
\begin{lstlisting}[style={verilog-style}]

module register #(parameter WIDTH=24) (input [WIDTH-1:0] in
, input clk,clr,enable, output reg [WIDTH-1:0] o);

	always@(posedge clk)

		if(enable)
			begin
				if(clr)
					o=0;
				else 
					o=in;
			end
\end{lstlisting}

\section{Código Controller}
\label{codecontroller}
\begin{lstlisting}[style={verilog-style}]

module controller(clk,reset,valid,valid2,selmul3_1,selmul3_2,seladd5_2,
seladd5_1,selr7,selsqrt,clr1,clr2,clr3,clr4,clr5,clr6,clr7,clr8,clr9,
clr10,clr11,enable1,enable2,enable3,enable4,enable5,enable6,enable7,
enable8,enable9,enable10,enable11,insqrt,inbottom,intop,
clrx1,clrx2,clrx3,clry1,clry2,clry3,enablex1,enablex2,enablex3,enabley1,
enabley2,enabley3,
selmul2_2,selmul2_1,selmul1_2,selmul1_1,seladd2_2,seladd1_1,seldiv_2,
seldiv_1,selr11,selr10,selr9,selr8,selr6,selr4,selr3,selr2,selr1,
seladd4_2,seladd4_1,seladd3_2,seladd3_1,seladd2_1,seladd1_2,
opadd1,opadd2,opadd3,opadd4,opadd5,
finish
);

//Parametros da maquina de estado
input  clk,reset,valid,valid2;
reg [5:0] cycle;
//Parametros de controle do  datapath
output  reg finish;
output reg selmul3_1,selmul3_2,seladd5_2,seladd5_1,
selr7,selsqrt,clr1,clr2,clr3,clr4,clr5,clr6,clr7,
clr8,clr9,clr10,clr11,enable1,enable2,enable3,
enable4,enable5,enable6,enable7,enable8,enable9,
enable10,enable11,insqrt,inbottom,intop;
output reg clrx1,clrx2,clrx3,clry1,clry2,clry3,enablex1,
enablex2,enablex3,enabley1,enabley2,enabley3;
output reg [1:0] selmul2_2,selmul2_1,selmul1_2,selmul1_1,
seladd2_2,seladd1_1,seldiv_2,seldiv_1,selr11,selr10,
selr9,selr8,selr6,selr4,selr3,selr2,selr1;
output reg [2:0] seladd4_2,seladd4_1,seladd3_2,seladd3_1,
eladd2_1,seladd1_2;
output reg [1:0] opadd1,opadd2,opadd3,opadd4,opadd5;


always @(*)           
begin
	case(cycle)
		6'b0:      
			begin          
				enablex1=1;
				enablex2=1;
				enablex3=1;
				enabley1=1;
				enabley2=1;
				enabley3=1;
				finish=0;
			end
		6'b000001:
			begin
				//disabled
				enable5=0;     
				enable6=0;   
				enable8=0;   
				enable9=0;   
				enable10=0;   
				enable11=0;     
				//ax1//                                      
				//add
				seladd1_2=3'b100;
				seladd1_1=2'b11;
				opadd1=2'b00;
				//register
				selr1=2'b10;
				clr1=0;
				enable1=1;
				//ay1//
				//add
				seladd2_2=2'b11;
				seladd2_1=3'b100;
				opadd2=2'b0;
				//register
				selr7=2'b01;
				clr7=0;
				enable7=1;
				//mx1//
				//mult
				selmul1_1=2'b10;
				selmul1_2=2'b10;
				//register
				selr2=2'b10;
				clr2=0;
				enable2=1;
				//mx2//
				//mult
				selmul2_1=2'b10;
				selmul2_2=2'b10;
				//register
				selr3=2'b10;
				clr3=0;
				enable3=1;
				//mx3//
				//mult
				selmul3_1=1'b1;
				selmul3_2=1'b1;
				//register
				selr4=2'b10;
				clr4=0;
				enable4=1; 
			end
		6'b000010:
			begin
				//disabled
				enable3=0;     
				enable4=0;   
				enable5=0;   
				enable6=0;   
				enable11=0;   
				//Mx//                                        
				//add
				seladd1_2=3'b011;
				seladd1_1=2'b10;
				opadd1=2'b01;
				//register
				clr1=0;
				enable1=1;
				selr1=2'b10;
				//My//
				//add
				seladd3_2=3'b101; //ok
				seladd3_1=3'b100; //ok
				opadd3=2'b01;//ok
				//register
				clr1=0;
				enable1=1;
				selr7=0;            
				//amx1//
				//add
				seladd2_2=2'b10;
				seladd2_1=3'b011;
				opadd2=2'b00;
				//register
				clr2=0;
				enable2=1;
				selr2=2'b00;
				//my1//
				//mult
				selmul1_1=2'b01;
				selmul1_2=2'b01;
				//register
				clr8=0;
				enable8=1;
				selr8=2'b10;
				//my2//
				//mult
				selmul2_1=2'b01;
				selmul2_2=2'b01;
				//register
				clr9=0;
				enable9=1;                 
				selr9=2'b10;
				//my3//
				//mult
				selmul3_1=0;
				selmul3_2=0;
				//register
				clr10=0;
				enable10=1;
				selr10=2'b11;                    
				end
				6'b000011:
				begin
				// disabled
				enable1=0;
				enable2=0;   
				enable3=0;   
				enable7=0;
				enable10=0;   
				enable11=0;   
				//Mx2
				//mult
				selmul1_2=2'b00;
				selmul1_1=2'b00;
				//register
				enable4=1;
				selr4=2'b01;
				//amy1
				//add
				seladd2_2=2'b01;
				seladd2_1=3'b010;
				opadd2=0;
				//register
				selr8=2'b01;
				clr8=0;
				enable8=1;
				//N1
				//add
				seladd4_2=3'b100;
				seladd4_1=2'b11;
				opadd4=2'b10;
				//register
				clr5=0;
				enable5=1;
				//Mx2
				//add
				seladd1_2=3'b010;
				seladd1_1=2'b01;
				opadd1=2'b01;
				//register 
				selr6=2'b10;
				clr6=0;
				enable6=1;
				//My2
				//mult
				selmul2_2=2'b00;
				selmul2_1=2'b00;
				//register
				selr9=2'b10;
				clr9=0;
				enable9=1;
			end
	6'b000100:
			begin
				enable1=0;
				enable2=0;   
				enable4=0;   
				enable5=0;
				enable6=0;   
				enable7=0;
				enable8=0;
				enable9=0;   
				enable11=0; 
				//Vx
				//add
				seladd3_2=3'b100;
				seladd3_1=3'b011;
				opadd3=2'b10;
				//register
				selr3=2'b01;
				clr3=0;
				enable3=1;
				//N2
				seladd4_2=3'b011;
				seladd4_1=2'b10;
				opadd4=10;
				//register
				selr4=2'b00;
				clr4=0;
				enable4=1; 
				//M2
				seladd1_2=3'b001;
				seladd1_1=2'b00;
				opadd1=2'b01;
				//register
				selr10=2'b10;
				clr10=0;
				enable10=1;
				end
				6'b000101:
				begin
				enable1=0;
				enable2=0;   
				enable4=0;   
				enable5=0;
				enable7=0;   
				enable8=0;
				enable10=0;   
				enable11=0;  
				//SDx
				//sqrt
				selsqrt=1;
				insqrt=1;
				//register
				selr6=2'b00;
				clr6=0;
				enable6=1;
				//N3
				//add
				seladd4_2=3'b010;
				seladd4_1=2'b01;
				opadd4=2'b10;
				//register
				selr3=2'b00;
				clr3=0;
				enable3=1;
				//Vy
				//add
				seladd3_2=3'b011;
				seladd3_1=3'b010;
				opadd3=2'b10;
				//register
				selr9=2'b00;
				clr9=0;
				enable9=1;
			end
		6'b000110:
			begin
				enable1=0;
				enable2=0;   
				enable3=0;
				enable4=0;   
				enable5=0;
				enable7=0;   
				enable8=0;
				enable10=0;   
				enable11=0;  
				//SDy
				//sqrt
				selsqrt=0;
				insqrt=1;
				//register
				selr9=2'b01;
				clr9=0;
				enable9=1;                                     
				end     
				6'b001101:
				begin
				enable1=0;
				enable3=0;
				enable4=0;   
				enable5=0;
				enable6=0;   
				enable7=0;   
				enable8=0;
				enable10=0;   
				enable11=0;
				//MSDx
				//add
				seladd2_2=2'b00;
				seladd2_1=3'b001; 
				opadd2=2'b10;
				//register
				selr2=2'b0;
				clr2=0;
				enable2=1;
				end
				6'b001110:
				
				begin
				enable3=0;
				enable4=0;   
				enable5=0;  
				enable7=0;
				enable9=0;
				enable10=0;   
				enable11=0;
				//DX1
				//add
				seladd1_2=3'b100;
				seladd1_1=2'b01;
				opadd1=2'b10;
				//register
				selr1=2'b10;
				clr1=0;
				enable1=1;
				//DX2
				//add
				seladd3_2=3'b010;
				seladd3_1=3'b001;
				opadd3=2'b10;
				//register
				selr2=2'b01;
				clr2=0;
				enable2=1;
				//DX3
				//add
				seladd4_2=3'b001;
				seladd4_1=2'b01;
				opadd4=2'b10;
				//register
				selr6=2'b01;
				clr6=0;
				enable6=1;                
				//MSDY
				//add
				seladd2_2=2'b01;
				seladd2_1=3'b000; 
				opadd2=2'b10;
				//register
				selr8=2'b01;
				clr8=0;
				enable8=1;  
			end
		6'b001111:
				begin
				enable1=0;
				enable2=0;
				enable3=0;
				enable4=0;   
				enable5=0;
				enable6=0;   
				enable7=0;    
				enable11=0;
				//DY
				//add
				seladd1_2=3'b000;
				seladd1_1=2'b00;
				opadd1=2'b10;
				//register
				selr8=2'b00;
				clr8=0;
				enable8=1;
				//DY2
				//add
				seladd3_2=3'b001;
				seladd3_1=3'b000;
				opadd3=2'b10;
				//register
				selr9=2'b00;
				clr9=0;
				enable9=1;
				//DY3
				//add
				seladd4_2=3'b010;
				seladd4_1=2'b00;
				opadd4=2'b10;
				//register
				selr10=2'b01;
				clr10=0;
				enable10=1;      
			end
		6'b010000:
				begin
				enable2=0;
				enable3=0;
				enable4=0;   
				enable5=0;
				enable7=0;
				enable8=0;
				enable9=0;    
				enable11=0;
				//D1
				//add
				seladd4_2=3'b0;
				seladd4_1=2'b0;
				opadd4=2'b0;
				//register
				selr1=2'b00;
				clr1=0;
				enable1=1;
				//D2
				//add
				seladd2_2=2'b01;
				seladd2_1=3'b011;
				opadd2=2'b0;
				//register
				selr6=2'b11;
				clr6=0;
				enable6=1;
				//D3
				//add
				seladd3_2=3'b000;
				seladd3_1=3'b011;
				opadd3=2'b0;
				//register
				selr10=2'b00;
				clr10=0;
				enable10=1;
				end 
				6'b010001:
				begin
				enable10=0;
				enable1=0;
				//Q1   
				inbottom=1;
				intop=1;
				//div
				seldiv_2=2'b10;
				seldiv_1=2'b10;
				//register
				selr11=2'b10;
				clr11=0;
				enable11=1;  
			end
		6'b010010:
			begin
				//Q2
				inbottom=1;
				intop=1;
				//div
				seldiv_2=2'b01;
				seldiv_1=2'b01;
			end
		6'b010011:
			begin
				//Q3
				inbottom=1;
				intop=1;
				//div
				seldiv_2=2'b00;
				seldiv_1=2'b00;
			end
		6'b100111:
			begin
				//aQ1 
				//add
				seladd5_2=1;
				seladd5_1=1;
				opadd5=2'b00;
				//register
				selr11=2'b00;
				clr11=0;
				enable11=1;                    
				end
				6'b101000:
				begin 
				//aQ2
				seladd5_2=1;
				seladd5_1=1;
				opadd5=2'b01; 
				//register 
				selr11=2'b01;
				clr11=0;   
				enable11=1;                               
			end
		6'b101001:
			begin
				//Verc
				seladd5_2=0;
				seladd5_1=0;
				opadd5=2'b10; 
				//register 
				selr11=2'b01;
				clr11=0;   
				enable11=1;  
				end
				6'b101010:
				begin   
				finish =1;
				clr1=1; 
				clr2=1; 
				clr3=1; 
				clr4=1; 
				clr5=1; 
				clr6=1;                        
				clr7=1;
				clr8=1;
				clr9=1;
				clr10=1;
				clr11=1;    
			end         
	endcase
end

always@(posedge clk, negedge reset)

	if(~reset)
		cycle <= 6'b0; 
	else   
			case(cycle)
				6'b0:
					cycle <= 6'b000001;
				6'b000001:
					cycle <= 6'b000010;
				6'b000010:
					cycle <= 6'b000011;
				6'b000011:   
					cycle <= 6'b000100;
				6'b000100:
					cycle <= 6'b000101;
				6'b000101:
					cycle <= 6'b000110;
				6'b000110:
					if(valid)
						cycle <= 6'b001101;
								 6'b001101:
								if(valid)
					cycle <= 6'b001110;
				6'b001110:
					cycle <= 6'b001111;
				6'b001111:
					cycle <= 6'b010000;
				6'b010000: 
					cycle <= 6'b010001;
				6'b010001:
					cycle <= 6'b010010;
				6'b010010:
					cycle <= 6'b010011;
				6'b010011:
						if(valid2)
					cycle <= 6'b100111;
				6'b100111:    
					cycle <= 6'b101000;
				6'b101000:            
					cycle <= 6'b101001;
				6'b101001:
					cycle <= 6'b101010;
				6'b101010:
					cycle <= 6'b00000; 
		endcase
endmodule
\end{lstlisting}


