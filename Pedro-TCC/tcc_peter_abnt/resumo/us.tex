% resumo em inglês
\begin{resumo}[Abstract]
 \begin{otherlanguage*}{english}
 	 Intrusion detection systems are increasingly necessary to ensure the security of services on the internet, as threats on the network have been developing more and more. DDoS-type attacks are very common these days, since there are enough resources to perform this type of attack in real time. Although there are solutions in software to detect DDoS attacks, but these are inefficient to perform for real time detection. In this work, a soft ip core was implemented for FPGAs that performs DDoS attack detection in real time, in a time of less than 1 $\mu$s. In addition, the implemented module combines low utilization of resources, because it makes use of fixed-point arithmetic, with a high precision when compared to the implementation in software with floating point. The module was implemented at RTL level and synthesized in a Xilinx 7-Series Artix FPGA.
   \noindent 
  
   \textbf{Key-words}: Network Security, Real-time, Module,
   Hardware, FPGA, Correlation. 
 \end{otherlanguage*}
\end{resumo}