\chapter{Referencial Teórico}
\label{CAP2}
\section{Ataques}
Ataques à uma rede de computadores sãos ações maliciosas em que softwares são utilizados para de alguma forma prejudicar, interromper uma ação ou invadir uma rede ou máquina, afim de se beneficiar com tal ação. Todavia o ataque, deve ser caracterizado dessa forma uma vez que se comprova que ele não é apenas ameaça e sim um agente malicioso. Por isso, vale ressaltar que o ataque é a ação propriamente dita, entretanto, uma ameaça à um sistema é algo que possa afetar ou atingir o seu funcionamento, operação, disponibilidade e integridade. Podemos dizer que um ataque ocorre quando uma ameaça intencional é realizada {1}.  Outra definição de ataque seria relacionada à seria o tráfego não desejado, que seria qualquer tipo de tráfego de rede não requisitado e/ou inesperado, cujo único propósito é consumir recursos computacionais da rede, desperdiçar tempo e dinheiro dos usuários e empresas e que pode gerar algum tipo de vantagem ou benefício (lucro) para seus criadores {2}. 
\section{DoS}
Um dos tipos mais comuns de ataque é o ataque DoS (Denial Of Service), que  é um tipo de ataque no qual uma ou mais máquinas buscam afetar uma vítima e tentam evitar que a vítima faça algum tipo de trabalho “útil” {3}. Uma característica importante desses tipos de ataques é que o objetivo principal desse tipo de ataque não é a invasão em busca de dados ou informações, mas a negação do serviço que a vítima está utilizando ou oferecendo.  Uma das estratégias mais utilizadas para a realização desse tipo de ataque é o envio de múltiplas requisições a vítima em questão, de forma que ela não suporta tantas requisições e começa a negar serviços, que antes ela era capaz de realizar {4}.

\section{DDoS}
Segundo {4} DDoS (Distributed Denial of Service) é um tipo de ataque DoS de grandes dimensões, ou seja, que utiliza até milhares de computadores para atacar uma determinada máquina, distribuindo a ação entre elas. Trata-se de uma forma muito utilizada, já que é o tipo de ataque mais comum na internet. Outra definifição seria "Um ataque DDoS usa muitos computadores para lançar um ataque DoS coordenado contra um ou mais alvos. Usando a tecnologia cliente / servidor, o atacante é capaz de multiplicar a eficácia do DoS significativamente, aproveitando os recursos de vários computadores cúmplices involuntários, que servem como plataformas de ataque "{5}.

\section{Detecção}
A detecção de um ataque DDoS é um trabalho relativamente complexo, uma vez que esse tipo de ataque ocorre em tempo real, e muitas vezes é difícil ser identificado a máquina em que realmente está sendo o atacante principal, por isso utilizamos o conceito de “tráfego não desejado”. Como visto anteriormente, antes da ação do ataque acontecer, existem as ameaças aquela rede ou máquina, por isso tráfegos que não desejados (que podem ser vistos como ameaça anteriormente a ação do ataque) devem ser identificados e o sistema de segurança podem tomar as devidas medidas. Segundo o {6} que é a referência de nosso trabalho, várias técnicas de detecção de ataques DDoS já foram propostas, muitas na área da estatística, porém essas soluções não proporcionam alta precisão de detecção DDoS ao usar um pequeno conjunto de recursos de tráfego, ou seja, com poucos dados de tráfego a detecção se torna inviável.

\section{Correlação}

Uma medida estatística que é muito utilizado em detecções em geral, seria a correlação, que é uma medida que mede o relacionamento entre duas variáveis (similaridade, linearidade e direção).  Uma das maiores vantagens da correlação é que ela não necessita de uma grande quantidade de variáveis para chegar a uma conclusão de relacionamento entre dois conjuntos. Portanto para uma detecção de ataques consistentes é necessária uma medida de correlação efetiva para classificar ataques DDoS  em tempo real mesmo quando usa um pequeno número de recursos de trafego. Uma correlação de forma resumida, é um conjunto de cálculos que a partir das variáveis de entrada, retorna o relacionamento entre as variáveis em um valor entre -1 e 1, então para ela ser efetivada basta se ter os valores de entrada e um módulo que realize esses cálculos e retorne o resultado da correlação. Segue abaixo a fórmula da correlação proposta pelo trabalho para ataques DDoS.

 \begin{equation}
 NaHiD(X,Y) = 1 - \frac{1}{n} \sum_{i=1}^{n} \frac{\left(|X(i) -	 Y(i)|\right)}{||\mu{X} - sX| - X(i)| + ||\mu{Y} - sY| - Y(i)|}
 \end{equation}
 onde
 \begin{itemize}
 	\item $\mu{X}$: Média aritmética do objeto de tráfego X.
 	\item $\mu{X}$: Média aritmética do objeto de tráfego Y.
 	\item $sX$: Desvio padrão do objeto de tráfego X.
 	\item $sY$: Desvio Padrão do objeto de tráfego Y.
 \end{itemize}

\section{Soluções em Hardware}
			
Quando é necessário realizar um considerável número de cálculos em um dado sistema, pode-se utilizar duas abordagens soluções utilizando software e soluções utilizando hardware. As utilizações de soluções em softwares possuem duas grandes vantagens: Podem ser utilizadas para soluções de propósitos gerais e são bem precisas em seus resultados. Já as soluções em hardware podem acumular uma lógica grande e complexa, além  de ter um alto desempenho {6}.  Na abordagem de Ataques DDoS, utilizamos um sistema em tempo real, que precisa de velocidade e uma relativa precisão, como os softwares em geral exigem grande quantidade de ciclos de CPU (“É o período de tempo no qual um computador lê e processa uma instrução em linguagem de máquina da sua memória ou a sequência de ações que a CPU realiza para executar cada instrução em código de máquina num programa”{7}), o que traz uma grande perca em sistemas de tempo real. Entretanto, umas arquiteturas baseadas em solução de hardware com algumas adaptações podem trazer desempenho e precisão uma vez que as medidas e ciclos são implementadas de acordo com o problema específico.

\section{Técnicas Utilizadas}
Nota-se que a correlação proposta por {6}, necessita de uma certa precisão, pois a mesma é uma medida de detecção. Para tal é necessário que os resultados de suas operações aritméticas tenham uma certa precisão. Basicamente seria necessário ver cada componente dessa formulação e realizar a implementação em hardware de forma acertiva, para que o sistema posso realizar conclusões e medidas precisas. Serão descritas as operações e as formas em hardware de realizar esses cálculos:
	
	\subsection{Soma:} A soma pode ser implementada, com componentes do tipo somador, ja bem conhecidos pela comunidade. Vale ressaltar que a soma pode ser utilizada no módulo, média e desvio padrão.
	\subsection{Módulo:} O módulo pode ser facilmente implementador em hardware com componentes do tipo somador. Vale ressaltar que o módulo pode ser utilizado no desvio padrão.
	\subsection{Divisão:} A divisão de dois números quaisquer possuem uma certa complexidade em Hardware, por não utilizar em suas operações apenas o conjunto dos números naturais, tendo que utilizar os conjuntos dos números fracionários. Para isso do utilizado um IP da plataforma VIVADO da Xinlix, chamado Divider Generator, esse IP realiza possue uma implementação de divisão de dois números e retorna um resultado em uma representação de números fracionados com uma precisão escolhida de forma personalizada \{Divider Generator v5.1 LogiCORE IP Product Guide\}.Esse IP configura a latência (quantos ciclos de clock, são necessários para um termino de operação) da computação da de divisão.
	\subsubsection{Ponto Fixo} 	
	Para representação de números fracionários, tem-se duas abordagens mais conhecidas: Ponto fixo e Ponto Flutuante. A representação em ponto flutuante, divide a palavra binária em duas partes (parte que representa o número inteiro e parte que representa o número decimal), essa divisão é feita dinamicamente, ou seja, uma hora a representação do número inteira tem um certo tamanho, outra hora pode ter outro tamanho, podendo ter uma maior precisão ou não de acordo com o a especificação do problema. Já a representação em ponto fixo divide a palavra binária em duas partes (parte que representa o número inteiro e parte que representa em decimal), porém com a divisão fixa , deixando a precisão sempre estática. A vantagem da representação em ponto fixo é a velocidade no qual essa representação é computada frente a representação em ponto flutuante, gerando um bom desempenho e uma certa precisão nos resultados {8}, o que para um problema em tempo real é essencial.
	
 	\subsection{Média aritmética:}  A média aritmética é um tipo de soma de variáveis seguido por um divisão, como foi dito uma divisão de dois números em hardware não é algo trivial. Por isso, uma alternativa é utilizar nessa divisão aritmética binária, de forma que com pequenos ajustes em uma operação (ajustes que não influenciem no resultado dessa operação),gere um resultado aproximado. No caso da média , seriam uma operação de soma de todos os elementos em questão, divido pelo número de elementos, segundo {4} a divisão por números pares são aproximadas do resultado apenas pelo deslocamento de algum bit, desprezando a parte fracionada. Vale ressaltar que a Média Aritmética pode ser utilizado no desvio padrão.
	\subsection {Desvio padrão:} O Desvio padrão é a variável em questão subtraída da média, após isso elevado ao quadrado. Após essa série de cálculos, calcula-se a raiz quadrada, tendo então a o desvio padrão. Então é necessário um multiplicador para realizar esse potenciação e uma raíz quadrada. Um multiplicador é facilmente implementado em hardware, pois existem componentes do tipo multiplicador. Já a raiz quadrada sua computação é um pouco mais complexa,por isso é necessário a utilização de um Ip core que faça o calculo de uma raiz. Para isso pode-se utilizar um  IP da plataforma VIVADO da Xinlix, chamado CORDIC {CORDIC v6.0} , que realiza  implementação de raízes quadradas. Basicamente esse IP recebe uma variável como entrada e em sua saída retorna a raiz aproximada do número de entrada .Esse IP configura a latência (quantos ciclos de clock, são necessários para um termino de operação) da computação  de raiz.
										
\section{FPGA}
Uma solução em Hardware que possui certas especificações, são totalmente voltadas parar sistemas embarcados. E quanto maior a especificação de um dado problema, será mais viável a escolha de que tipo de sistema embarcado que se deve utilizar para acomodar a lógica do problema em questão. Como foi pontuado acima o problema da detecção de ataques em tempo real, requer duas características principais desempenho e precisão. Para realizar esse detecção poderíamos utilizar dois tipos de unidade de processamento mais conhecidas ASICs(Circutos integrados para uma aplicação específica) e as FPGAS(Arranjo de Portas Programáveis em Campo). Porém segundo {6}, as FPGAs oferecem adaptabilidade dinâmica, que é importante para aplicações que requerem mudanças freqüentes em suas configurações, como a detecção de ataques DDoS que evoluem com freqüência. Além disso segundo {9} O tempo de projetos baseados em FPGA é normalmente conhecido com muita precisão, o que é essencial para um sistema em tempo real.

Para o módulo de correlação em hardware receber as instâncias de tráfego no trabalho {6} foi implementado  um módulo  chamado de pré-processador e ao final da computação o módulo de hardware envia para outro módulo que irá realizar algum tratamento no sistema, chamado de gerenciador de segurança. Nota-se que os módulos do pré-processador e do gerenciador de segurança são implementados separadamente usando software. As máquinas que implementam esses módulos e o FPGA podem comunicar usando as interfaces de E / S de alta velocidade suportadas pelos FPGAs modernos, como PCI e Ethernet. O módulo de detecção de ataque recebe a instância	de tráfego do módulo pré-processador. Além disso, ele recebe o perfil normal e um valor limiar do banco de dados do perfil criado pelo gerenciador de segurança. Cada uma das instâncias de tráfego e o perfil normal são vetores que consistem em três recursos de tráfego. O módulo de detecção de ataque calcula primeiro o NaHiD VERC entre a instância de tráfego de entrada e o perfil normal. O valor de correlação calculado é comparado com o limite para classificar a ocorrência de tráfego recebido como ataque ou normal. O resultado da classificação é armazenado no banco de dados Log para análise off-line pelo gerenciador de segurança. Além disso, um alarme é gerado no caso de a instância ser classificada como um ataque.


\section{FPGA'S Xinlix de série 7}
As FPGAs da série Xilinx 7 compreendem quatro famílias de FPGA que abordam uma gama completa de requisitos nos sistemas como: baixo custo, pequenos tamanhos, sensíveis ao custo, aplicações de alto volume de largura para altas conectividades, capacidade lógica e capacidade de processamento de sinal para aplicações de alto desempenho {10}.

Artix (Baixo custo) , Kintex (Balanço de alta performance com baixo custo), Virtex (Sistemas de alta performance) e ZYNQ (Aplicações em sistemas embarcados em geral).
A diferença entre essas famílias está basicamente em sua performance e custo , consequentemente para qual finalidade elas deverão ser usadas. Pra corroborar com isso nota que cada família tem variações em relação máxima capacidade dos recursos que ambas possuem, como Células Lógicas, Blocos de RAM , pinos I/O e etc .
Cada bloco lógico configurável (CLB) é composto por dois Slices que podem ser do tipo M (SLICEM) , podem ser utilizados para memória e lógica, ou do tipo L (SLICEL), podem ser utilizados apenas para lógica e aritmética. 
Cada Slice pode ter como recursos gerenciáveis 4 LUTs(Look-up Tables) de 6 entradas , multiplexadores , carry chains e 4 flip-flops/latches.
As interfaces I/O, em geral, trabalham pra proporcionar altas velocidades de resposta sem tentar perder integridade no sinal. Além de serem projetadas para diferentes padrões (Voltagens, larguras e protocolos). Nas famílias da 7 series elas podem ser caracterizados por dois tipos : High Range (HR), suportando padrões I/O de tensões de ate 3.3 V,  e High Performance (HP), suportando somente padrões I/O de tensões de até 1.8V e projetado para alta performance, por isso dependente da família.
Todos os membros das famílias de série 7 tem o mesmo bloco RAM/FIFO(First in first out) , operação totalmente síncrona, muitas opções de configurações (true dual port, simple dual-port, single-port) e etc {11}. 




