\chapter[Conclusões e Trabalhos Futuros]{Conclusões e Trabalhos Futuros}


Foi construído um módulo em hardware que se mostrou capaz de detectar em tempo real, uma vez que foi comparado com um trabalho já validado e com soluções em softwares e obteve comportamento similar ou melhor que essas soluções.

Como podemos ver na tabela, \ref{Tab:TD} temos um ganho de tempo, significativo do módulo em hardware em relação a solução em \textit{software}. Além disso não temos uma precisão com erros pequenos, como podemos ver na tabela \ref{Tab:Tb} . 

Como podemos ver nas tabelas \ref{Tab:RUAC}, \ref{Tab:RUP}, \ref{Tab:TD} e \ref{Tab:Tb} temos um ganho, tanto de utilização minima de componentes como no tempo de detecção. Vale ressaltar que a \textit{FPGA} que esse trabalho utiliza é a \textit{Artix Série 7} , como vimos no \ref{CAP2} possui baixa potência. Já o artigo de comparação usa uma \textit{Virtex Série 5}, que possui alta performance. Mesmo assim, os resultados desse trabalho foram melhores.

Esses resultados foram alcançados devido a estudos prévios de aritmética de números binários,  e a implementação dos mesmos. Além disso foram utilizados métodos de otimização de tempo nas  operações aritméticas.
	
Além disso foram utilizados componentes, da IDE vivado da Xinlinx, que foram essências para a realização de detecção, pois as operações que esses ip's implementam, são complexas e custosas. Além disso existe uma maior confiabilidade nos resultados, mediante a credibilidade da ferramenta utilizada.